\chapter{Parallel method for a congruence by generating pairs}
\label{chap:pairs}

A congruence is a binary relation, and is therefore formally described as a set
of pairs.  In a computational setting, it is rarely practical to keep track of
every pair in a congruence; a congruence on a semigroup of size $n$ contains
$n^2$ pairs in the worst case, and on an infinite semigroup contains an infinite
number of pairs.  A congruence can be described in more concise ways utilising
known theory: for example, taking advantage of its being an equivalence relation
and recording only its equivalence classes; or in the case of a Rees congruence,
storing a generating set for the ideal which defines it.  A variety of different
ways to describe a congruence are explained in Chapter \ref{chap:converting},
along with ways to convert one to another.  However, a congruence is still just
a set of pairs, and by reducing the number of pairs we store, we can describe a
congruence very concisely using them.

A congruence $\rho$ is \textit{generated by} a set of pairs $R$ if it consists
of only the pairs in $R$ along with the pairs required by the axioms of a
congruence (reflexivity, symmetry, transitivity and compatibility).  Thus a
congruence can be described completely by storing only a few pairs.
Indeed, experiments on a range of interesting semigroups in Chapter
\ref{chap:lattice} show that many congruences can be generated by an extremely
small number of generating pairs, and indeed that most of the congruences
studied are principal (generated by a single pair). %TODO: actually do this

Another justification for the use of generating pairs is that it is a completely
generic representation.  Some special types of semigroup have their own abstract
representations of congruences---for inverse semigroups, one can study
kernel-trace pairs; for groups, normal subgroups; for completely simple or
completely 0-simple semigroups, linked triples---but generating pairs can
represent a congruence on any semigroup whatsoever.  Furthermore, a researcher
might be interested in what pairs are implied by a given pair or set of pairs in
a congruence, and this representation can answer such questions.

Left congruences and right congruences can also be described using generating
pairs, and some algorithms designed for two-sided congruences can be used with
minor modifications to compute information about left and right congruences.

This chapter describes a parallelised approach for computing a congruence from a
set of generating pairs, as implemented in \texttt{libsemigroups}
\cite{libsemigroups}.  First we will give a general outline of the system
and what questions it hopes to answer, then we will describe in detail each
algorithms used, its advantages and disadvantages, and when it can be applied.
Finally we will explain how the different algorithms are executed together, and
consider their implementation in \cite{libsemigroups}.

\section{Why Parallelisation?}

Parallel processing has seen major advances in the last ten years, with
multi-core processors becoming the norm in many types of computers, and
processors with 4, 8, or even 16 cores becoming common on a desktop PC.  This
being the case, it is desirable to parallelise mathematical algorithms wherever
possible, and take advantage of the ability to execute multiple
threads of instructions concurrently.  Some algorithms, such as simple binary
searches, are ``embarrassingly parallel''---that is, they can be split into
independent threads which require almost no communication with each other.
These are suited so well to parallelisation that splitting the operation into
$n$ parallel threads reduces the expected run-time to barely more than
$\frac{1}{n}$ what it would be when run in a single thread.  Other
algorithms do not parallelise so well: sometimes threads have to communicate, or
use shared resources, causing significant slowdown and severely limiting the
improvements which can be made by parallelising.

When it comes to computing information about a congruence from generating pairs,
there are various different approaches which can be taken: in Sections
\ref{sec:tc}, \ref{sec:kbfp} and \ref{sec:p}, we describe three
possible algorithms.  Depending on what sort of semigroup is described, several
or all of these might be appropriate.  However, depending on certain properties
of the congruence, one might perform far better than another.  For example, the
pair orbit algorithm works well on congruences which contain few non-reflexive
pairs, while the Todd-Coxeter algorithm tends to work well on congruences with
few classes (i.e.~very many pairs).  Given only a set of generating pairs, these
properties may be unknown in advance, which makes it difficult to choose a good
algorithm.

The natural answer to this problem is the core concept of this chapter: a
parallel approach which does not attempt to parallelise individual algorithms,
but which runs all known algorithms at the same time, each in a different
thread, and simply halts all threads as soon as any one completes.  Since these
algorithms do not interact with each other in any way, the total run-time will
be close to the minimum run-time of all the different algorithms.  This is
particularly important, since some of the algorithms never terminate for certain
semigroups and congruences (for example, those of infinite size).

\section{Applicable types of semigroup}

The type of element in a semigroup affects which methods will be most effective,
or even which methods will be applicable.  For this purpose, we divide
semigroups into two categories: finitely presented semigroups, and
\textit{concrete} semigroups.  By \textit{concrete}, we mean a finite semigroup
whose elements can be multiplied and compared quickly, without reference to the
semigroup as a whole; these could be semigroups of transformations, partial
permutations, bipartitions, matrices, or other finite objects.  Their elements
will be known in advance.  Finitely presented semigroups are a different case,
in that their elements are not known in advance, and indeed it may not be known
whether or not they are finite.  Finitely presented monoids are treated as
equivalent to finitely presented semigroups, since a semigroup presentation can
be attained by simply adding a generator for the identity and relations to make
it multiplicatively neutral.

Decidability?

\section{Questions}

Each method we are about to explain can provide a variety of information, but it
is important to consider which questions we aim to answer.  Our system should be
able to return the following information about a given congruence when
requested:

\begin{itemize}
\item Presence of a given pair $(x,y)$
\item Number of congruence classes
\item The elements in each non-trivial congruence class
\item Class number of a given element
\end{itemize}

Note that for finitely presented semigroups, these questions may be
undecidable.  In the case that a question is decidable, our approach should
return an answer in finite time.

\section{Finding a presentation}

Where do we get the presentation from in the first place?

\section{The methods}

\subsection{Pair orbit enumeration}
\label{sec:p}

Background

The P algorithm

Using Knuth-Bendix: KBP

Semigroups/congs it works/works best on

Complexity

\subsection{Todd-Coxeter}
\label{sec:tc}

The Todd-Coxeter algorithm, was originally described in 1936 in
\cite{todd_coxeter_1936}.  It was an algorithm to enumerate the cosets of a
finitely generated subgroup of a finitely presented group.  Arriving before the
advent of electronic computers, the algorithm was originally intended to be
carried out by hand.  Perhaps the earliest automatic implementation was on the
EDSAC II computer in Cambridge \cite{leech_1963}.  Since then, a wide variety of
efficient, optimised versions have been implemented, for example \cite{ace}.

A variation of Todd-Coxeter for semigroups was described in 1967
\cite{neumann_1967}.  The algorithm takes a presentation $\pres X R$
for a semigroup $S$ and computes the right regular representation of $S^1$ with
respect to the generators $X$---that is, it computes all the elements of $S$ and
the result of right-multiplying each element by each generator.  Since the
original algorithm makes very little use of those properties unique to groups,
the method applied to semigroups is essentially the same.  Other descriptions of
Todd-Coxeter for semigroups can be found in \cite[Chapter 12]{ruskuc_thesis} and
\cite[Chapter 1.2]{walker_thesis}, and a variation specific to inverse
semigroups can be found in \cite{cutting_thesis}.  Our version of the algorithm
is based closely on an implementation by G\"otz Pfeiffer, found in
\cite[\texttt{lib/tcsemi.gi}]{gap}, and based on \cite{walker_thesis}.

We will now describe the Todd-Coxeter method as used in the context of this
chapter.  Though this does not represent significant new work, it is an
important part of the overall parallel approach, and in order to understand its
uses and limitations, it is described here in full.

\subsubsection{Setup}

The Todd-Coxeter algorithm is based on a table, where each row corresponds to a
single congruence class (or equivalently, a single element of the quotient
semigroup).  The columns of the table correspond to the generators of the
semigroup, and the entry in row $i$, column $j$ represents the element found by
taking element $i$ and right-multiplying it by generator $j$.  These entries may
be blank, and two different rows may be found to describe the same element.
Mathematically, we can view this table as a triple $(n, \mathbf{N}, \tau)$
consisting of:
\begin{itemize}
\item an integer $n \in \mathbb{N}$ representing the number of rows in the table;
\item a set $\mathbf{N} \subseteq \{1, \dots, n\}$ containing the indices of the
  \textit{undeleted} rows; and
\item a function $\tau: \mathbf{N} \times X \to \mathbf{N} \cup \{0\}$, where
  $(i, x)\tau$ is equal to the entry in row $i$ and the column corresponding to
  generator $x$---with $0$ representing a blank entry.
\end{itemize}

Suppose that we have a semigroup presentation $\pres X R$ for a semigroup
$S$.  The table is initialised with a single row, numbered $1$.  This row
corresponds to the empty word $\varepsilon$, or the adjoined identity of the
monoid $S^1$.  The row is empty, containing a blank entry in all $|X|$ columns.
In our mathematical notation, we define $n=1$ and
$(1,x)\tau = 0$ for all $x \in X$.

We can naturally extend the function
$\tau: \mathbf{N} \times X \to \mathbf{N} \cup \{0\}$
to a function 
$\bar{\tau}: \mathbf{N} \times X^* \to \mathbf{N} \cup \{0\}$
which is described as follows.
If $w \in X^*$ and $w=w_1 \dots w_n$, where $w_1, \dots, w_n \in X$,
then we can define $\bar\tau$ recursively by
$$
(i, w)\bar\tau = \left\{
\begin{matrix*}[l]
  i & \textnormal{if~} w=\varepsilon,\\
  0 & \textnormal{if~} (i, w_1)\tau=0,\\
  ((i, w_1)\tau, w_2 \dots w_n)\bar\tau & \textnormal{otherwise}.
\end{matrix*} \right.
$$

The effect of $\bar\tau$ is to trace an entire word through the table, starting
at a given row.

\subsubsection{Elementary operations}

We now describe 3 operations which may be applied to the table.  These
operations will be described loosely to give an intuition behind what they are
designed to do, and then each one will be described formally in pseudo-code
(Algorithms \ref{alg:add}, \ref{alg:trace} and \ref{alg:coinc}).  We will then
describe the overall Todd-Coxeter procedure which uses these operations
(Algorithm \ref{alg:tc}).

\begin{itemize}
\item \textsc{Add}: Fill in a blank entry and add a row to the table;
\item \textsc{Trace}: Trace a relation from a row;
\item \textsc{Coinc}: Process a coincidence.
\end{itemize}

The first operation, \textsc{Add}, is simple.  A new row is added at the bottom
of the table, and its number is written into the blank cell in the table
specified by the arguments given to \textsc{Add}.  Pseudo-code for this
operation is given in Algorithm \ref{alg:add}.

\begin{algorithm}
\caption{The \textsc{Add} algorithm}
\label{alg:add}
\begin{algorithmic}[1]
\Procedure{Add}{$i, x$}
\State $n \gets n + 1$
\State $\mathbf{N} \gets \mathbf{N} \cup \{n\}$
\For{$x \in X$}
  \State $(n, x)\tau := 0$
\EndFor
\State $(i, x)\tau \gets n$  
\EndProcedure
\end{algorithmic}
\end{algorithm}

\textsc{Trace} takes two arguments: a relation $v=w$ from $R$, and a row $e$
in the table.  We now give an informal description of this procedure---see
Algorithm \ref{alg:trace} for pseudo-code.

We start from row $e$, and find $(e, v)\bar\tau$ by processing $v$ one letter at a time: we
find the column corresponding to the letter, and look at that position in the
table; this gives us the number of the new row which we move to.  If we
encounter a blank entry before the final letter, we apply \textsc{Add} to that
cell, and follow the new entry.  At the end of the process we have an entry in
the table, blank or filled, called $(e, v)\bar\tau$.  We repeat the
process with the other word to find $(e, w)\bar\tau$.

To satisfy $v=w$, we need to set these two entries such that
$(e, v)\bar\tau = (e, w)\bar\tau$.
\begin{itemize}
\item If $(e, v)\bar\tau$ and $(e, w)\bar\tau$ are both blank, then we apply
  \textsc{Add} to $(e, v)\bar\tau$ and copy the entry into $(e, w)\bar\tau$.
\item If just one of the entries is filled, then the filled entry is copied into
  the blank one.
\item If both entries are filled and equal, we need do nothing.
\item If both entries are filled and are distinct, we apply \textsc{Coinc} to
  the two entries.
\end{itemize}

\begin{algorithm}
\caption{The \textsc{Trace} algorithm}
\label{alg:trace}
\begin{algorithmic}[1]
\Procedure{Trace}{$e, v = w$}
\State Write $v = v_1 \dots v_m$ \Comment $(v_i \in X \text{~for~} 1 \leq i \leq m)$
\State Write $w = w_1 \dots w_n$ \Comment $(w_i \in X \text{~for~} 1 \leq i \leq n)$
\State $s \gets e$
\For{$i \in \{1, \dots, m-1\}$}
  \If{$(s, v_i)\tau = \varepsilon$}
    \State \Call{Add}{$s, v_i$}
  \EndIf
  \State $s \gets (s, v_i)\tau$
\EndFor
\State $t \gets e$
\For{$i \in \{1, \dots, n-1\}$}
  \If{$(t, w_i)\tau = \varepsilon$}
    \State \Call{Add}{$t, w_i$}
  \EndIf
  \State $t \gets (t, w_i)\tau$
\EndFor

\If{$(s, v_m)\tau = (t, w_n)\tau = \varepsilon$}
  \State \Call{Add}{$s, v_m$}
  \State $(t, w_n)\tau \gets (s, v_m)\tau$
\ElsIf{$(s, v_m)\tau = \varepsilon$}
  \State $(s, v_m)\tau \gets (t, w_n)\tau$
\ElsIf{$(t, w_n)\tau = \varepsilon$}
  \State $(t, w_n)\tau \gets (s, v_m)\tau$
\ElsIf{$(s, v_m)\tau \neq (t, w_n)\tau$}
  \State \Call{Coinc}{$(s, v_m)\tau, (t, w_n)\tau$}
\EndIf

\EndProcedure
\end{algorithmic}
\end{algorithm}

\textsc{Coinc} is used when two rows in the table are found to refer to the same
element of $S$.  The higher-numbered row is deleted, and all occurrences of the
higher number in the table are replaced by the lower number.  The two rows are
combined into one, with all known information being preserved.  This may imply
that another pair of rows are equal, another coincidence that should be
processed by a recursive call to \textsc{Coinc} (or should be stored in a list
for processing after this call is finished).

\begin{algorithm}
\caption{The \textsc{Coinc} algorithm}
\label{alg:coinc}
\begin{algorithmic}[1]
\Require $r < s$
\Procedure{Coinc}{$r, s$}
\State $\mathbf{N} \gets \mathbf{N} \setminus \{s\}$
\For{$e \in \mathbf{N}$}
  \For{$x \in X$}
    \If{$(e, x)\tau = s$}
      \State $(e, x)\tau \gets r$
    \EndIf
  \EndFor
\EndFor
\For{$x \in X$}
  \If{$(r, x)\tau = 0$}
    \State $(r, x)\tau \gets (s, x)\tau$
  \ElsIf{$(r, x)\tau \neq (s, x)\tau \textbf{~and~} (s, x)\tau \neq 0$}
    \State \Call{Coinc}{$(r, x)\tau, (s, x)\tau$}
  \EndIf
\EndFor
\EndProcedure
\end{algorithmic}
\end{algorithm}

Now that we have these three operations, it is simple to describe the overall
algorithm.  We go through the list of used rows, starting with row $1$.  To each
of these rows we apply each relation from $R$, using \textsc{Trace}.  Each call
to \textsc{Trace} may, of course, invoke calls to \textsc{Add} and
\textsc{Coinc}, so rows will be appended to the list as the algorithm
progresses.  When the end of the list is reached, the table should be a Cayley
graph for the finitely presented semigroup (excluding the identity row $1$).
Note that the end of the list is not guaranteed to be reached, for example in
the case of an infinite semigroup whose table would grow indefinitely.  In this
case, the procedure would run forever.

\begin{algorithm}
\caption{The \textsc{Todd-Coxeter} algorithm for semigroups}
\label{alg:tc}
\begin{algorithmic}[1]
\Procedure{Todd-Coxeter}{$\pres X R$}
\State $n := 1$
\State $\mathbf{N} := \{1\}$
\For{$x \in X$}
  \State $(1, x)\tau := 0$
\EndFor
\For{$e \in \mathbf{N}$}
  \For{$(u=v) \in R$}
    \State \Call{Trace}{$e, u=v$}
  \EndFor
\EndFor
\EndProcedure
\end{algorithmic}
\end{algorithm}

\subsubsection{An example}
We now give an example of the Todd-Coxeter algorithm running on the semigroup
presentation
$$\pres{a, b}{ba=ab, b^2=b, a^3=ab, a^2b=a^2}.$$
We initialise the table to look like Table \ref{tab:tc1}.
\tctableAB{tab:tc1}
{Initial position}
{ 1 & & \\ }

The list $\mathbf{N}$ of undeleted rows contains only a single entry, $1$.  We
begin by tracing each relation on the row $1$, starting with $ba=ab$.  This
relation makes us call \textsc{Add} on the cell $(1, b)$, creating a new row,
$2$, which is added to $\mathbf{N}$.  Similarly, we must call \textsc{Add} on
the cell $(1, a)$, creating a row $3$.  At the end of the \textsc{Trace}, we
must set $(1, ba)\bar\tau$ equal to $(1, ab)\bar\tau$, so we set both
$(2, a)\tau$ and $(3, b)\tau$ to $4$ (as in Table \ref{tab:tc2}).
\tctableAB{tab:tc2}
{Position after \textsc{Trace}($1, ba=ab$)}
{
  1 & 3 & 2 \\
  \cline{2-3}
  2 & 4 & \\
  \cline{2-3}
  3 & & 4 \\
  \cline{2-3}
  4 & & \\
}
Next, we apply \textsc{Trace}($1, b^2=b$).  Since $(1, b)\bar\tau$ is already
set, we just set $(1, b^2)\bar\tau$ equal to it: $(2, b)\tau \gets 2$.  See
Table \ref{tab:tc3}.
\tctableAB{tab:tc3}
{Position after \textsc{Trace}($1, b^2=b$)}
{
  1 & 3 & 2 \\
  \cline{2-3}
  2 & 4 & 2 \\
  \cline{2-3}
  3 & & 4 \\
  \cline{2-3}
  4 & & \\
}
Still on row $1$, we apply \textsc{Trace} to the third relation, $a^3=ab$.  This
creates a new row for $(1, a^2)\bar\tau = (3, a)\tau = 5$.  The new row's $a$
entry is set to be the same as $(1, ab)\bar\tau$, which is $4$
(see Table \ref{tab:tc4}).
\tctableAB{tab:tc4}
{Position after \textsc{Trace}($1, a^3=ab$)}
{
  1 & 3 & 2 \\
  \cline{2-3}
  2 & 4 & 2 \\
  \cline{2-3}
  3 & 5 & 4 \\
  \cline{2-3}
  4 & & \\
  \cline{2-3}
  5 & 4 & \\
}

The final relation for row $1$ is $a^2b=a^2$.
$(1, a^2b)\bar\tau$ is currently blank, and is set to the current value of
$(1, a^2)\bar\tau$, which is $5$.  Hence $(5, b)\tau \gets 5$
(as in Table \ref{tab:tc5}).
\tctableAB{tab:tc5}
{Position after \textsc{Trace}($1, a^2b=a^2$)}
{
  1 & 3 & 2 \\
  \cline{2-3}
  2 & 4 & 2 \\
  \cline{2-3}
  3 & 5 & 4 \\
  \cline{2-3}
  4 & & \\
  \cline{2-3}
  5 & 4 & 5 \\
}
We have now finished with row $1$, and we proceed to the next row in
$\mathbf{N}$, which is $2$.  Accordingly, we apply the first relation,
\textsc{Trace}($2, ba=ab$).  The value of $(2, ba)\bar\tau$ is $4$, whereas the
value of $(2, ab)\bar\tau$ has not yet been set.
We set it by applying $(4, b)\tau \gets 4$.
See Table \ref{tab:tc6}.
\tctableAB{tab:tc6}
{Position after \textsc{Trace}($2, ba=ab$)}
{
  1 & 3 & 2 \\
  \cline{2-3}
  2 & 4 & 2 \\
  \cline{2-3}
  3 & 5 & 4 \\
  \cline{2-3}
  4 & & 4 \\
  \cline{2-3}
  5 & 4 & 5 \\
}
Proceeding with \textsc{Trace}($2, b^2=b$), we find that
$(2, b^2)\bar\tau = (2, b)\bar\tau$ already, so we make no modifications to the
table.  Next, \textsc{Trace}($2, a^3=ab$) discovers that $(2, a^2)\bar\tau$ is
not set, and so we call \textsc{Add}($4, a$), creating a new row $6$.
Now $(6, a)\tau$ is set to $(2, ab)\bar\tau$ which is equal to $4$.
See Table \ref{tab:tc7}.
\tctableAB{tab:tc7}
{Position after \textsc{Trace}($2, a^3=ab$)}
{
  1 & 3 & 2 \\
  \cline{2-3}
  2 & 4 & 2 \\
  \cline{2-3}
  3 & 5 & 4 \\
  \cline{2-3}
  4 & 6 & 4 \\
  \cline{2-3}
  5 & 4 & 5 \\
  \cline{2-3}
  6 & 4 & \\
}
The final relation for row $2$ is $a^2b=a^2$, setting $(6, b)\tau \gets 6$ (see
Table \ref{tab:tc8}).
\tctableAB{tab:tc8}
{Position after all relations on row $2$}
{
  1 & 3 & 2 \\
  \cline{2-3}
  2 & 4 & 2 \\
  \cline{2-3}
  3 & 5 & 4 \\
  \cline{2-3}
  4 & 6 & 4 \\
  \cline{2-3}
  5 & 4 & 5 \\
  \cline{2-3}
  6 & 4 & 6 \\
}
Next we move onto row $3$, and we apply \textsc{Trace}($3, ba=ab$).  Inspecting
the table shows $(3, ba)\bar\tau = 6$ but $(3, ab)\bar\tau = 5$, giving rise to
a coincidence.  We apply \textsc{Coinc}($5, 6$), which deletes row $6$, rewrites
any occurrences of $6$ in the table to $5$, and copies row $6$ into row $5$
(yielding no new information).  The result is shown in Table \ref{tab:tc9}.  The
rest of the relations are applied to row $3$, and to the remaining rows in the
table, but no changes are made to the table, so Table \ref{tab:tc9} is the final
state.
\tctableAB{tab:tc9}
{Final position}
{
  1 & 3 & 2 \\
  \cline{2-3}
  2 & 4 & 2 \\
  \cline{2-3}
  3 & 5 & 4 \\
  \cline{2-3}
  4 & \cancel{\textcolor{gray}{6}}5\!\!\! & 4 \\
  \cline{2-3}
  5 & 4 & 5 \\
  \cline{2-3}
  \textcolor{gray}{6} & \textcolor{gray}{4} & \textcolor{gray}{6} \\[-1.6ex]
  \hline\noalign{\vspace{\dimexpr 1.4ex}} \cline{2-3}
}

We can now delete row $1$, which acts as an appended identity, and we find a
description of the semigroup's multiplication, with relation to its generators.
This description can be represented as a Cayley graph,
as shown in Figure \ref{fig:tc-cayley-graph}.
\begin{figure}[H]
  \centering
  \begin{dot2tex}
    //dot
    digraph {
      rankdir=LR
      node [shape=circle]
      2
      3
      4
      5
      2 -> 4 [label=a]
      2 -> 2 [label=b]
      3 -> 5 [label=a]
      3 -> 4 [label=b, dir=both]
      4 -> 5 [label=a, dir=both]
      4 -> 4 [label=b]
      5 -> 5 [label=b]
    }
  \end{dot2tex}
  \caption{Right Cayley graph of $\pres{a, b}{ba=ab, b^2=b, a^3=ab, a^2b=a^2}$}
  \label{fig:tc-cayley-graph}
\end{figure}
It is worth noting that the columns of the table now give a right representation
of $S$.  That is, $S$ is isomorphic to the semigroup generated by the
transformations $\transV 34554$ and $\transV 22445$.

\subsubsection{Improvements}
Left/right congruences

Pre-filling the table

Semigroups/congs it works/works best on

Complexity

\subsubsection{Implementation}

Rows will be added to the table, and deleted from it.  A list must be kept of
rows which are in use; when a row is added, its position in the table should be
appended to this list at the end, and when a row is deleted it should be removed
from its position in the list and added to a list of ``free rows'' which can be
reused later.  The ``rows in use'' list is best implemented as a doubly-linked
list, so that single entries can be added and removed with as little processor
work as possible.

\subsection{Knuth-Bendix and Froidure-Pin}
\label{sec:kbfp}

Background

The KBFP algorithm

Semigroups/congs it works/works best on

Complexity

\section{Running in parallel}

How do we tie together all the different algorithms?

\section{Implementation}

Practical considerations in libsemigroups

Showing off speed

Drawbacks
