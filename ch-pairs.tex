\chapter{Parallel method for a congruence by generating pairs}
\label{chap:pairs}

A congruence is a binary relation, and is therefore formally described as a set
of pairs.  In a computational setting, it is rarely practical to keep track of
every pair in a congruence---a congruence on a semigroup of size $n$ contains
$n^2$ pairs in the worst case, and on an infinite semigroup contains an infinite
number of pairs.  A congruence can be described in more concise ways utilising
known theory: for example, taking advantage of its being an equivalence relation
and recording only its equivalence classes; or in the case of a Rees congruence,
storing a generating set for the ideal which defines it.  A variety of different
ways to describe a congruence are described in Chapter \ref{chap:converting},
along with ways to convert one to another.  However, a congruence is still just
a set of pairs, and by reducing the number of pairs we store, we can describe a
congruence very concisely using them.

A congruence $\rho$ is \textit{generated by} a set of pairs $R$ if it consists
of only the pairs in $R$ along with the pairs required by the axioms of a
congruence (reflexivity, symmetry, transitivity and compatibility).  Thus a
congruence can be described completely by storing only a few pairs.
Indeed, experiments on a range of interesting semigroups in Chapter
\ref{chap:lattice} show that many congruences can be generated by an extremely
small number of generating pairs, and indeed that most of the congruences
studied are principal (generated by a single pair). %TODO: actually do this

Another justification for the use of generating pairs is that it is a completely
generic representation.  Some special types of semigroup have their own abstract
representations of congruences---for inverse semigroups, one can study
kernel-trace pairs; for groups, normal subgroups; for completely simple or
completely 0-simple semigroups, linked triples---but generating pairs can
represent a congruence on any semigroup whatsoever.  Furthermore, a researcher
might be interested in what pairs are implied by a given pair or set of pairs in
a congruence, and this representation can answer such questions.

Left congruences and right congruences can also be described using generating
pairs, and some algorithms designed for two-sided congruences can be used with
minor modifications to compute information about left and right congruences.

This chapter describes a parallelised approach for computing a congruence from a
set of generating pairs, as implemented in \texttt{libsemigroups}
\cite{libsemigroups}.  First we will give a the general outline of the system
and what questions it hopes to answer, then we will describe in detail each
algorithms used, its advantages and disadvantages, and when it can be applied.
Finally we will explain how the different algorithms are executed together, and
consider their implementation in \cite{libsemigroups}.

\section{Computation}

Parallel processing is great

Among algorithms, some are better than others, but we don't know in advance.

Explain the basic idea.

\section{Types of semigroups}

Concrete vs f.p.

Monoids are trivially different.

\section{Todd-Coxeter}
\label{sec:tc}

Background

The TC algorithm

Pre-filling the table

Semigroups/congs it works/works best on

Complexity

\section{KBFP}
\label{sec:kbfp}

Background

The KBFP algorithm

Semigroups/congs it works/works best on

Complexity

\section{Pair orbit enumeration}
\label{sec:p}

Background

The P algorithm

Using Knuth-Bendix: KBP

Semigroups/congs it works/works best on

Complexity

\section{Running in parallel}

How do we tie together all the different algorithms?

\section{Implementation}

Practical considerations in libsemigroups

Showing off speed

Drawbacks
