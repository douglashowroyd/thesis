% Text operators
\DeclareMathOperator\rank{rank}
\DeclareMathOperator\Cong{Cong}
\DeclareMathOperator\id{id}
\DeclareMathOperator\im{im}
\DeclareMathOperator\tr{tr}
\DeclareMathOperator\dom{dom}
\DeclareMathOperator\codom{codom}
\DeclareMathOperator\coker{coker}

% Green's relations
\newcommand{\HH}{\mathrel{\mathscr{H}}}
\newcommand{\LL}{\mathrel{\mathscr{L}}}
\newcommand{\RR}{\mathrel{\mathscr{R}}}
\newcommand{\DD}{\mathrel{\mathscr{D}}}
\newcommand{\JJ}{\mathrel{\mathscr{J}}}

\newcommand{\hH}{\mathscr{H}}
\newcommand{\lL}{\mathscr{L}}
\newcommand{\rR}{\mathscr{R}}
\newcommand{\dD}{\mathscr{D}}
\newcommand{\jJ}{\mathscr{J}}

\newcommand{\nHH}{\mathrel{\cancel{\mathscr{H}}}}
\newcommand{\nLL}{\mathrel{\cancel{\mathscr{L}}}}
\newcommand{\nRR}{\mathrel{\cancel{\mathscr{R}}}}
\newcommand{\nDD}{\mathrel{\cancel{\mathscr{D}}}}
\newcommand{\nJJ}{\mathrel{\cancel{\mathscr{J}}}}

% Other relations
\newcommand{\sS}{\mathcal{S}}
\newcommand{\tT}{\mathcal{T}}
\newcommand{\R}{\mathbf{R}}
\newcommand{\Rs}{\mathbf{R}^\sharp}
\newcommand{\po}{\mathbf{O}}

% Special semigroups
\newcommand{\PT}{\mathcal{P\!T\!}}     % Partial transformations
\newcommand{\T}{\mathcal{T}}           % Transformations
\newcommand{\I}{\mathcal{I}}           % Partial perms
\newcommand{\PO}{\mathcal{P\!O\!}}     % Order-preserving partial transformations
\newcommand{\OO}{\mathcal{O}}          % Order-preserving transformations
\newcommand{\POI}{\mathcal{P\!O\!I\!}} % Order-preserving partial perms
\newcommand{\LZ}{\mathcal{L\!Z}}
\newcommand{\RZ}{\mathcal{R\!Z}}
\newcommand{\Z}{\mathcal{Z}}
\newcommand{\PBR}{\mathcal{P}\!B}
\newcommand{\PP}{\mathscr{P}\!\mathcal{P}}
\newcommand{\PB}{\mathcal{P\!B}}
\newcommand{\Brau}{\mathcal{B}}
\newcommand{\Jon}{\mathcal{J}}
\newcommand{\Sym}{\mathcal{S}}
\newcommand{\Alt}{\mathcal{A}}
\newcommand{\Mot}{\mathcal{M}}
\newcommand{\Prt}{\mathcal{P}}
\newcommand{\Cyc}{\mathcal{C}}

% Other little shortcuts
\newcommand{\bn}{\mathbf{n}}

% Rewriting systems
\newcommand{\rws}{\mathbf{R}}
\newcommand{\tostar}{\stackrel{*}{\to}}
\newcommand{\lrstar}{\stackrel{*}{\leftrightarrow}}

% Double angle brackets
\newcommand{\llangle}{\langle\!\langle}
\newcommand{\rrangle}{\rangle\!\rangle}

% Ticks and crosses
\newcommand{\cmark}{\ding{51}}
\newcommand{\xmark}{\ding{56}}

% Principal factors
\newcommand{\pf}[1]{\overline{#1}}

% Algorithmicx
\algnewcommand\True{\textbf{true}\space}
\algnewcommand\False{\textbf{false}\space}
\algnewcommand{\LComment}[1]{\State \(\triangleright\) \emph{#1}}
\algnewcommand{\Or}{~\textbf{or}~}
\algnewcommand{\And}{~\textbf{and}~}
\algnewcommand{\Continue}{\textbf{continue}}
\algnewcommand{\Goto}{\textbf{goto}~}

% Citing OEIS
\newcommand{\citeoeis}[1]{\cite[\href{https://oeis.org/#1}{#1}]{oeis}}

% Transformations
\newcommand{\transII}[2]{
\begin{pmatrix}
  1 & 2 \\
  #1 & #2
\end{pmatrix}
}
\newcommand{\transV}[5]{
\begin{pmatrix}
  1 & 2 & 3 & 4 & 5 \\
  #1 & #2 & #3 & #4 & #5
\end{pmatrix}
}
\newcommand{\transN}[3]{
\begin{pmatrix}
  1 & 2 & \ldots & n \\
  #1 & #2 & \ldots & #3
\end{pmatrix}
}

% Presentations
\newcommand{\pres}[2]{\left\langle\,#1 ~\middle|~ #2\,\right\rangle}

% Todd-Coxeter tables with 2 generators
\newcommand{\tctableAB}[4]{
\begin{table}[H]
  \centering
  \begin{tabular}{c | c | c |}
    \multicolumn{1}{c}{} &
    \multicolumn{1}{c}{$a$} &
    \multicolumn{1}{c}{$b$} \\
    \cline{2-3}
    #4
    \cline{2-3}
  \end{tabular}
  \caption[#2]{#3}
  \label{#1}
\end{table}
}

% Partition diagrams
% \tikzset{>=latex}
\newcommand{\bipartdiagscale}{.36}
\newcommand{\tv}[2]{ % Transversal line
  \draw(#1,2)--(#2,0);
}
\newcommand{\tc}[2]{ % Top curve
  \draw(#1, 1.875) .. controls (#1, 1.5) and (#2, 1.5) .. (#2, 1.875);
}
\newcommand{\bc}[2]{ % Bottom curve
  \draw(#1, 0.125) .. controls (#1, 0.5) and (#2, 0.5) .. (#2, 0.125);
}
\newcommand{\bC}[2]{ % Bottom curve (bigger)
  \draw(#1, 0.125) .. controls (#1, 0.75) and (#2, 0.75) .. (#2, 0.125);
}
\newcommand{\utv}[2]{ % Transversal (upper diagram)
  \draw(#1,4)--(#2,2);
}
\newcommand{\utc}[2]{ % Top curve (upper diagram)
  \draw(#1, 3.875) .. controls (#1, 3.5) and (#2, 3.5) .. (#2, 3.875);
}
\newcommand{\ubc}[2]{ % Bottom curve (upper diagram)
  \draw(#1, 2.125) .. controls (#1, 2.5) and (#2, 2.5) .. (#2, 2.125);
}
\newcommand{\uutv}[2]{ % Transversal (3rd diagram)
  \draw(#1,6)--(#2,4);
}
\newcommand{\uutc}[2]{ % Top curve (3rd diagram)
  \draw(#1, 5.875) .. controls (#1, 5.5) and (#2, 5.5) .. (#2, 5.875);
}
\newcommand{\uubc}[2]{ % Bottom curve (3rd diagram)
  \draw(#1, 4.125) .. controls (#1, 4.5) and (#2, 4.5) .. (#2, 4.125);
}
\newcommand{\uubC}[2]{ % Bottom curve (3rd diagram) (bigger)
  \draw(#1, 4.125) .. controls (#1, 4.75) and (#2, 4.75) .. (#2, 4.125);
}

\newcommand{\trans}[5]{ % Transformation
  \begin{tikzpicture}[scale=\bipartdiagscale, baseline={([yshift=-.8ex]current bounding box.center)}]
    \fill[gray!20] (0.8,0) rectangle (5.2,2);
    \fill(1,2)circle(.125); \fill(1,0)circle(.125); \draw(1,2)--(#1,0);
    \fill(2,2)circle(.125); \fill(2,0)circle(.125); \draw(2,2)--(#2,0);
    \fill(3,2)circle(.125); \fill(3,0)circle(.125); \draw(3,2)--(#3,0);
    \fill(4,2)circle(.125); \fill(4,0)circle(.125); \draw(4,2)--(#4,0);
    \fill(5,2)circle(.125); \fill(5,0)circle(.125); \draw(5,2)--(#5,0);
  \end{tikzpicture}
}

\newcommand{\bipartdiag}[1]{ % Single bipartition diagram
  \begin{tikzpicture}[scale=\bipartdiagscale, baseline={([yshift=-.8ex]current bounding box.center)}]
    \fill[gray!20] (0.8,0) rectangle (5.2,2);
    \fill(1,2)circle(.125); \fill(1,0)circle(.125);
    \fill(2,2)circle(.125); \fill(2,0)circle(.125);
    \fill(3,2)circle(.125); \fill(3,0)circle(.125);
    \fill(4,2)circle(.125); \fill(4,0)circle(.125);
    \fill(5,2)circle(.125); \fill(5,0)circle(.125);
    #1
  \end{tikzpicture}~
}

\newcommand{\bipartdiagII}[1]{ % Single bipartition diagram (degree 2)
  \begin{tikzpicture}[scale=\bipartdiagscale, baseline={([yshift=-.8ex]current bounding box.center)}]
    \fill[gray!20] (0.8,0) rectangle (2.2,2);
    \fill(1,2)circle(.125); \fill(1,0)circle(.125);
    \fill(2,2)circle(.125); \fill(2,0)circle(.125);
    #1
  \end{tikzpicture}~
}

\newcommand{\doublebipartdiag}[1]{ % Double diagram (to show multiplication)
  \begin{tikzpicture}[scale=\bipartdiagscale, baseline={([yshift=-.8ex]current bounding box.center)}]
    \fill[gray!20] (0.8,0) rectangle (5.2,4);
    \fill(1,4)circle(.125); \fill(1,2)circle(.125); \fill(1,0)circle(.125);
    \fill(2,4)circle(.125); \fill(2,2)circle(.125); \fill(3,0)circle(.125);
    \fill(3,4)circle(.125); \fill(3,2)circle(.125); \fill(2,0)circle(.125);
    \fill(4,4)circle(.125); \fill(4,2)circle(.125); \fill(5,0)circle(.125);
    \fill(5,4)circle(.125); \fill(5,2)circle(.125); \fill(4,0)circle(.125);
    #1
  \end{tikzpicture}~
}

\newcommand{\triplebipartdiag}[1]{ % Triple diagram (to show multiplication)
  \begin{tikzpicture}[scale=\bipartdiagscale, baseline={([yshift=-.8ex]current bounding box.center)}]
    \fill[gray!20] (0.8,0) rectangle (5.2,6);
    \fill(1,6)circle(.125); \fill(1,4)circle(.125); \fill(1,2)circle(.125); \fill(1,0)circle(.125);
    \fill(2,6)circle(.125); \fill(2,4)circle(.125); \fill(2,2)circle(.125); \fill(2,0)circle(.125);
    \fill(3,6)circle(.125); \fill(3,4)circle(.125); \fill(3,2)circle(.125); \fill(3,0)circle(.125);
    \fill(4,6)circle(.125); \fill(4,4)circle(.125); \fill(4,2)circle(.125); \fill(4,0)circle(.125);
    \fill(5,6)circle(.125); \fill(5,4)circle(.125); \fill(5,2)circle(.125); \fill(5,0)circle(.125);
    #1
  \end{tikzpicture}~
}

\newcommand{\bipart}[4]{ % Bracketed notation for a bipartition
  \left[
    \hspace{-0.75mm}
    \scriptsize
    \doublespacing
    \arraycolsep=1.0mm
    \begin{array}{#1}
      #3 \\ \cline{#2}
      #4
    \end{array}
    \hspace{-0.6mm}
  \right]
}
\newcommand{\mc}{\multicolumn}
\newcommand{\two}[2]{\multicolumn{3}{c}{\begin{array}{c|c}#1\ \ &\ #2\end{array}}}
\newcommand{\bipartABCD}{\bipart{c|c|c|c|c|c}{4-6}
  {A_1 & \ldots & A_q & C_1 & \ldots & C_r}
  {B_1 & \ldots & B_q & D_1 & \ldots & D_s}}
