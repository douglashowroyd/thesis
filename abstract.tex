\thispagestyle{plain}
\begin{center}
    \large
    \textbf{Abstract}
\end{center}
\addcontentsline{toc}{chapter}{Abstract}


This body of work is based upon the following three papers that the author wrote during his PhD with Jonathan Fraser and Han Yu: \cite{fraser-howroyd2,howroyd-yu,howroyd}. 
\\ \\
We start by introducing many of the common tools and notation that will be used throughout this thesis. This will cover the main notions of dimensions discussed from both the set and the measure perspectives. An emphasis will be placed on their relationships where possible. This will provide a solid base upon which to expand and any concepts introduced in Chapter 1 will be relevant throughout the rest of this work. Many of the standard results in this part can be found in fractal geometry textbooks such as \cite{falconer, mattila} if further reading was desired.
\\ \\
The first results discussed in Chapter 2 will cover some of the regularity dimensions' properties such as general bounds in relation to the Assouad and lower dimensions, local dimensions and the $L^q$-spectrum. The Assoaud and lower dimensions are known to interact pleasantly with weak tangents and these ideas are discussed in the regularity dimension setting. We then calculate the regularity dimensions for several specific example measures such as self-similar and self-affine measures which provides an opportunity to discuss the sharpness of the previously obtained bounds. This work originates in \cite{fraser-howroyd2} where the upper regularity dimension was studied, with many of the lower regularity dimension results being natural extensions.
\\ \\
In Chapter 3 we continue the study of the upper and lower regularity dimensions with an emphasis on how they can be used to quantify doubling and uniform perfectness of measures. This starts with an explicit relation between the upper regularity dimension and the doubling constants along with a similar link between the lower regularity dimension and the constants of uniform perfectness. We then turn our attention to a technical result which can be made more quantitative thanks to the regularity dimensions. It is interesting to study how properties, such as doubling, change under pushforwards by different types of maps, here we study the regularity dimensions of pushforward measures with respect to quasisymmetric homeomorphisms. We round this chapter out with an interesting application of the lower regularity to Diophantine approximation by noting the equivalence between uniform perfectness and weakly absolutely $\alpha$-decaying measures. The original material for this part can be found in \cite{howroyd} with part of the first section integrating a result of \cite{fraser-howroyd2}. 
\\ \\
Finally, in Chapter 4, we will consider graphs of Brownian motion, and more generally, graphs of L\'evy processes. This will involve the calculation of the lower and Assouad dimensions for such sets and then the regularity dimensions of measures pushed onto these graphs from the real line. Such graphs are the only examples in this thesis for which the Assouad and lower dimensions had not been previously calculated so we delve deeper into the area, studying graphs of functions defined as stochastic integrals as well. This chapter is based on the paper \cite{howroyd-yu} for the set theoretic half, with the regularity dimension results coming from \cite{howroyd}.
