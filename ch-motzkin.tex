\chapter{Congruences of the Motzkin monoid}
\label{chap:motzkin}

In Chapter \ref{chap:lattice} we explained a relatively quick way of computing
all of a semigroup's congruences, along with information about how they fit into
their lattice structure.  This was implemented in the Semigroups package
\cite{semigroups}, greatly increasing the size and complexity of semigroups
whose congruence lattices can be found using a computer.

One of the first semigroups towards which this new methodology was directed was
the bipartition monoid $\Prt_n$, whose congruence lattice was not previously
known.  Computing this lattice for the first few values of $n$ showed a lattice
with a relatively simple structure, which did not appear to increase much in
complexity as $n$ grew higher than $3$.  The congruence lattices of various
submonoids of $\Prt_n$ were also computed, and appeared to have a similar
structure.  With the rapidly increasing size of $\Prt_n$ (see Table
\ref{tab:pn-size}) it proved impractical to na\"ively calculate the congruence
lattices beyond $n=4$, but careful study of the lattices for small values of
$n$, along with those lattices computed for various submonoids of $\Prt_n$,
yielded a general classification of the congruence lattice of $\Prt_n$ for
arbitrary $n$, along with a classification of the congruence lattices of various
important submonoids.  This classification is explained and proven in
\cite{ourpaper}.

In this chapter, we will examine the structure of these congruence lattices,
focusing in particular on the Motzkin monoid $\Mot_n$, which was the present
author's particular focus when contributing to \cite{ourpaper}.  We will start
with the definition of the Motzkin monoid, then describe some preliminary
theory, then describe the Motzkin monoid's lattice of congruences, and
finally give a brief description of how these ideas can be extended to $\Prt_n$
and its other submonoids.

\section{The Motzkin monoid $\Mot_n$}
\label{sec:motzkin-monoid}
In order to define the Motzkin monoid, we must first define a \textit{planar}
bipartition.

\begin{definition}
  \label{def:planar}
  \index{planar bipartition}
  A bipartition is called \textbf{planar} if it can be represented in diagram
  form with all edges contained inside the rectangle formed by the vertices, and
  without any edges crossing.
\end{definition}

\begin{example}
  Let $\alpha = \bipart{c|c|c|c}{2-4}{1,2 &3 &4 &5}{2,5 &1 &\mc2{c}{3,4}}$ and
  $\beta = \bipart{c|c|c|c}{3-4}{2 &5 &1,3 &4}{1 &3,4 &2 &5}$.  As can be seen
  in Figure \ref{fig:planar}, $\alpha$ is planar.  However, $\beta$ cannot be
  drawn inside the rectangle without the upper block $\{1,3\}$ crossing lines
  with the transversal $\{2, 1'\}$---hence, $\beta$ is not planar.
\end{example}

\begin{figure}[h]
  \centering
  $$\alpha = \bipartdiag{\tc12\tv22\bC25 \bc34} \qquad
  \beta = \bipartdiag{\tc13 \tv21 \tv54\bc34}$$
  \caption{A planar and a non-planar bipartition}
  \label{fig:planar}
\end{figure}

We can now define the Motzkin monoid.

\begin{definition}
  \label{def:motzkin}
  \index{Motzkin monoid}
  \nomenclature[Mn]{$\Mot_n$}{Motzkin monoid}
  The \textbf{Motzkin monoid} $\Mot_n$ is the submonoid of $\Prt_n$ consisting
  of all planar bipartitions of degree $n$ in which every block has size $1$ or
  $2$.
\end{definition}

To see that this is indeed a monoid, we should observe that it is closed.  It is
easy to see that the product of two planar bipartitions is also planar, since a
double diagram as in Figure \ref{fig:bipartition-example} would contain no
crossing lines, and therefore would resolve to a product with no crossing lines.
It is also easy to see that if two bipartitions have no block larger than $2$,
their product also has no block larger than $2$: any transversal can only
contain one point in $\bn$ and one point in $\bn'$, so any transversal in the
product can only contain two points; the upper and lower blocks of the product
are inherited from the original bipartitions, so they will not break the
condition either.

The Motzkin monoid $\Mot_n$ grows much slower than its parent $\Prt_n$, having
only $\sum_{k=0}^n \binom{2n}{2k}C_k$ elements \cite[A026945]{oeis}, where $C_k$
is the $k$th Catalan number.  Its size in comparison with $\Prt_n$ is
shown in Table \ref{tab:mn-size}.

\begin{table}[h]
  \centering
  \renewcommand\arraystretch{1.0}
  \begin{tabular}{| r | r | r |}
    \hline
    $n$ & $|\Mot_n|$ & $|\Prt_n|$ \\
    \hline
     1 &           2 &                  2 \\
     2 &           9 &                 15 \\
     3 &          51 &                203 \\
     4 &         323 &              4 140 \\
     5 &       2 188 &            115 975 \\
     6 &      15 511 &          4 213 597 \\
     7 &     113 634 &        190 899 322 \\
     8 &     853 467 &     10 480 142 147 \\
     9 &   6 536 382 &    682 076 806 159 \\
    10 &  50 852 019 & 51 724 158 235 372 \\
    \hline
  \end{tabular}
  \renewcommand\arraystretch{0.7}
  \caption{Sizes of $\Mot_n$ and $\Prt_n$ for small values of $n$}
  \label{tab:mn-size}
\end{table}

The Motzkin monoid $\Mot_n$ shares a number of features with $\Prt_n$---indeed,
we will see later that its congruence lattice is very similar.  Like $\Prt_n$,
$\Mot_n$ is regular One important
similarity is in its Green's relations.  Consider the following proposition,
akin to Proposition \ref{prop:bipartition-greens}.

\begin{proposition}
  \label{prop:mn-greens}
  Let $\alpha$ and $\beta$ be bipartitions in $\Mot_n$.  The following hold:
  \begin{enumerate}[\rm(i)]
  \item $\alpha \RR \beta$ if and only if $\dom \alpha = \dom \beta$ and
    $\ker \alpha = \ker \beta$;
  \item $\alpha \LL \beta$ if and only if $\codom \alpha = \codom \beta$ and
    $\coker \alpha = \coker \beta$;
  \item $\alpha \JJ \beta$ if and only if $\rank \alpha = \rank \beta$;
  \item $J_\alpha \leq J_\beta$ if and only if $\rank \alpha \leq \rank \beta$;
  \item the ideals of $\Mot_n$ are precisely the sets
    $I_r=\{\alpha \in \Mot_n : \rank \alpha \leq r\}$ for
    $r \in \{0, \ldots, n\}$.
  \end{enumerate}
  \begin{proof}
    For (i) to (iii), see \cite[Theorem 2.4]{deg_motzkin}.  For (iv) and (v),
    see \cite[Proposition 2.6]{deg_motzkin}.
  \end{proof}
\end{proposition}

This description of the Motzkin monoid's Green's relations, and its containment
of $\JJ$-classes and ideals, will help us greatly later on.  However, one
consequence of (i) and (ii) gives $\Mot_n$ a feature which $\Prt_n$ does not
share, namely the following corollary.

\begin{corollary}
  \label{cor:mn-h-trivial}
  The Motzkin monoid $\Mot_n$ is $\HH$-trivial.
  \begin{proof}
    Let $\alpha, \beta \in \Mot_n$ such that $\alpha \HH \beta$.  This tells us
    that $\alpha \LL \beta$ and $\alpha \RR \beta$, so by Proposition
    \ref{prop:mn-greens} parts (i) and (ii), we know that $\alpha$ and $\beta$
    share the same domain, kernel, codomain and cokernel.  The upper blocks and
    lower blocks of $\alpha$ and $\beta$ must certainly be the same, since they
    are just the blocks of the kernel and cokernel that do not lie in the domain
    or codomain.  The only choice is in the transversals: which blocks in the
    domain connect to which blocks in the codomain.  In $\Prt_n$ there are
    $(\rank \alpha)!$ ways of choosing this match-up; but in $\Mot_n$ there is
    only one way possible, since we cannot allow any lines in the diagram to
    cross over.  Hence $\alpha = \beta$.
  \end{proof}
\end{corollary}


\section{Preliminaries}
\label{sec:motzkin-prelim}
We will start by defining some concepts which allow us to find certain
congruences in any semigroup: \textit{retractable ideals} (Definition
\ref{def:retractable-ideal}) and \textit{lifting congruences} (Definition
\ref{def:lifting-congruence}).  It will turn out that all the congruences of
$\Mot_n$ can be built using these two building blocks.

\begin{definition}
  \label{def:retractable-ideal}
  \index{retractable ideal} \index{retraction}
  Let $S$ be a semigroup with a minimal ideal $M$.  An ideal $I$ of $S$ is
  \textbf{retractable} if there exists a homomorphism $f: I \to M$ such that
  $xf = x$ for all $x \in M$; such a homomorphism is called a
  \textbf{retraction}.
\end{definition}

\begin{definition}
  \label{def:lifting-congruence}
  \index{lifting congruence}
  \nomenclature{$\xi$}{Lifting congruence}
  Let $S$ be a semigroup with a minimal ideal $M$.  A congruence $\xi$ on $M$ is
  a \textbf{lifting congruence} if any, and hence all, of the following
  equivalent conditions are satisfied:
  \begin{enumerate}[\rm(i)]
  \item $\Delta_S \cup \xi$ is a congruence on $S$;
  \item there exists a congruence $\zeta$ on $S$ such that
    $\xi=\zeta\cap(M\times M)$;
  \item for all $(x,y) \in \xi$ and $s \in S$, $(xs,ys),(sx,sy) \in \xi$.
  \end{enumerate}
\end{definition}

In order to use these building blocks to produce new congruences, we first need
to establish some results about them.  Note that, since $\Mot_n$ is finite, it
always has a minimal ideal.  More specifically, the minimal ideal of $\Mot_n$ is
given by $I_0 = \{\alpha \in \Mot_n : \rank \alpha = 0\}$ (see Proposition
\ref{prop:mn-greens}).  The following lemma will be used at various times
throughout this chapter.

\begin{lemma}
  \label{lem:retract-aux}
  Let $S$ be a semigroup with a regular minimal ideal $M$, and let $I$ be an
  ideal of $S$. If $f: I\rightarrow M$ is a retraction, then $(sxt)f=s(xf)t$ for
  all $x\in I$ and $s,t\in S^1$.
  \begin{proof}
    We will prove the lemma by showing that $(sx)f=s(xf)$ for any $s\in S ^ 1$
    and $x\in I$.  The equality $(xt)f=(xf)t$ is dual, and then
    $(sxt)f=s(xt)f=s(xf)t$.  Let $e\in M$ be any right identity for $xf$; such
    an~$e$ exists because $M$ is regular. Then, since $f$ is a retraction and
    $e,xe\in M$, we have $xf=(xf)e=(xf)(ef)=(xe)f=xe$.  Next, let $e_1\in M$ be
    a left identity for $(sx)f$.  Then
    $(sx)fe=e_1(sx)fe=(e_1f)(sx)f(ef)=(e_1sxe)f
     =(e_1s)f(xf)(ef)=(e_1s)f(xf)=(e_1sx)f=(e_1f)(sx)f=e_1(sx)f=(sx)f$.
    So $e$ is a right identity for $(sx)f$ as well, and hence $(sx)f=sxe=s(xf)$,
    as required.
  \end{proof}
\end{lemma}

This gives rise to an important result which we can use later when we combine
retractable ideals with lifting congruences.  Note first that, since $\Mot_n$ is
a regular semigroup, its minimal ideal is also regular.

\begin{corollary}
  \label{cor:retract-unique}
  Let $S$ be a semigroup with a regular minimal ideal $M$.  If $I$ is a
  retractable ideal of $S$, then there exists a unique retraction from $I$ to
  $M$.
  \begin{proof}
    Suppose $f,g:I\to M$ are retractions, and let $x\in I$.  Let $e\in M$ be a
    left identity for $xf$.  Using Lemma \ref{lem:retract-aux}, we see that
    $xf = e (xf) = (ex)f = ex = (ex)g = e (xg)$.  A dual argument shows that
    $xg=(xf) e'$ for some $e'$.  But then $xf = e (xg) = e (xf) e' = (xf)e'=xg$.
  \end{proof}
\end{corollary}

The effect of Corollary \ref{cor:retract-unique} is that, for a semigroup with a
regular minimal ideal, we can talk about \textit{the} retraction of a
retractable ideal without any loss of generality.  This result is the last thing
we need to use our two building blocks to produce a new congruence: a
\textit{lifted congruence}.

\begin{definition}
  \label{def:lifted-congruence}
  \index{lifted congruence}
  Let $S$ be a semigroup with a minimal ideal $M$, let $I$ be a retractable
  ideal of $S$, and let $\xi$ be a lifting congruence on $M$.  We associate to
  the pair $(I,\xi)$ the relation
  $$\zeta_{I,\xi}= \Delta_S \cup \{(x,y) \in I \times I : (xf,yf) \in \xi\},$$
  where $f:I \to M$ is the unique retraction of $I$.
  We call $\zeta_{I,\xi}$ the \textbf{lifted congruence} of $(I,\xi)$.
\end{definition}

In order to justify the name \textit{lifted congruence}, we require the
following proposition.

\begin{proposition}\label{prop:lift}
  The relation $\zeta_{I,\xi}$ in Definition \ref{def:lifted-congruence} is a
  congruence on $S$.
  \begin{proof}
    Write $\zeta=\zeta_{I,\xi}$ for brevity, and let $f:I\to M$ be the
    retraction.  Let $(x,y)\in\zeta$ and $s\in S$ be arbitrary.  We must show
    that $(xs,ys),(sx,sy)\in\zeta$.  This is clear if $x=y$, so suppose
    $x,y\in I$ and $xf\mathrel\xi yf$.  Since~$I$ is an ideal, we have
    $xs,ys\in I$.  By Lemma \ref{lem:retract-aux}, and condition (iii) of
    Definition \ref{def:lifting-congruence}, we have
    $(xs)f = (xf)s \mathrel\xi (yf)s = (ys)f$, showing that $(xs,ys)\in\zeta$.
    A dual argument shows that $(sx,sy)\in\zeta$.
\end{proof}
\end{proposition}

This construction now gives us a usable source of congruences.  All that is
required is to find lifting congruences and retractable ideals of a semigroup,
and a number of new congruences can be described.  It turns out that this is an
excellent source of congruences for $\Mot_n$, yielding every congruence on the
semigroup, as we will see later.

\section{Congruence lattice of $\Mot_n$}
\label{sec:motzkin-congs}


\section{Other monoids}
\label{sec:motzkin-other}
