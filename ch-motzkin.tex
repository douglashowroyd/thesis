\chapter{Congruences of the Motzkin monoid}
\label{chap:motzkin}

In Chapter \ref{chap:lattice} we explained a relatively quick way of computing
all of a semigroup's congruences, along with information about how they fit into
their lattice structure.  This was implemented in the Semigroups package
\cite{semigroups}, greatly increasing the size and complexity of semigroups
whose congruence lattices can be found using a computer.

One of the first semigroups towards which this new methodology was directed was
the bipartition monoid $\Prt_n$, whose congruence lattice was not previously
known.  Computing this lattice for the first few values of $n$ showed a lattice
with a relatively simple structure, which did not appear to increase much in
complexity as $n$ grew higher than $3$.  The congruence lattices of various
submonoids of $\Prt_n$ were also computed, and appeared to have a similar
structure.  With the rapidly increasing size of $\Prt_n$ (see Table
\ref{tab:pn-size}) it proved impractical to na\"ively calculate the congruence
lattices beyond $n=4$, but careful study of the lattices for small values of
$n$, along with those lattices computed for various submonoids of $\Prt_n$,
yielded a general classification of the congruence lattice of $\Prt_n$ for
arbitrary $n$, along with a classification of the congruence lattices of various
important submonoids.  This classification is explained and proven in
\cite{ourpaper}.

In this chapter, we will examine the structure of these congruence lattices,
focusing in particular on the Motzkin monoid $\Mot_n$, which was the present
author's particular focus when contributing to \cite{ourpaper}.  We will start
with the definition of the Motzkin monoid, then describe some preliminary
theory, then describe the Motzkin monoid's lattice of congruences, and
finally give a brief description of how these ideas can be extended to $\Prt_n$
and its other submonoids.

\section{The Motzkin monoid $\Mot_n$}
\label{sec:motzkin-monoid}
In order to define the Motzkin monoid, we must first define a \textit{planar}
bipartition.

\begin{definition}
  \label{def:planar}
  A bipartition is called \textbf{planar} if it can be represented in diagram
  form with all edges contained inside the rectangle formed by the vertices, and
  without any edges crossing.
\end{definition}

\begin{example}
  Let $\alpha = \bipart{c|c|c|c}{2-4}{1,2 &3 &4 &5}{2,5 &1 &\mc2{c}{3,4}}$ and
  $\beta = \bipart{c|c|c|c}{3-4}{2 &5 &1,3 &4}{1 &3,4 &2 &5}$.  As can be seen
  in Figure \ref{fig:planar}, $\alpha$ is planar.  However, $\beta$ cannot be
  drawn inside the rectangle without the upper block $\{1,3\}$ crossing lines
  with the transversal $\{2, 1'\}$---hence, $\beta$ is not planar.
\end{example}

\begin{figure}[h]
  \centering
  $$\alpha = \bipartdiag{\tc12\tv22\bC25 \bc34} \qquad
  \beta = \bipartdiag{\tc13 \tv21 \tv54\bc34}$$
  \caption{A planar and a non-planar bipartition}
  \label{fig:planar}
\end{figure}

We can now define the Motzkin monoid.

\begin{definition}
  The \textbf{Motzkin monoid} $\Mot_n$ is the submonoid of $\Prt_n$ consisting
  of all planar bipartitions of degree $n$ in which every block has size $1$ or
  $2$.
\end{definition}

To see that this is indeed a monoid, we should observe that it is closed.  It is
easy to see that the product of two planar bipartitions is also planar, since a
double diagram as in Figure \ref{fig:bipartition-example} would contain no
crossing lines, and therefore would resolve to a product with no crossing lines.
It is also easy to see that if two bipartitions have no block larger than $2$,
their product also has no block larger than $2$: any transversal can only
contain one point in $\bn$ and one point in $\bn'$, so any transversal in the
product can only contain two points; the upper and lower blocks of the product
are inherited from the original bipartitions, so they will not break the
condition either.

The Motzkin monoid $\Mot_n$ grows much slower than its parent $\Prt_n$, having
only $\sum_{k=0}^n \binom{2n}{2k}C_k$ elements \cite[A026945]{oeis}, where $C_k$
is the $k$th Catalan number.  Its size in comparison with $\Prt_n$ is
shown in Table \ref{tab:mn-size}.

\begin{table}[h]
  \centering
  \renewcommand\arraystretch{1.0}
  \begin{tabular}{| r | r | r |}
    \hline
    $n$ & $|\Mot_n|$ & $|\Prt_n|$ \\
    \hline
     1 &           2 &                  2 \\
     2 &           9 &                 15 \\
     3 &          51 &                203 \\
     4 &         323 &              4 140 \\
     5 &       2 188 &            115 975 \\
     6 &      15 511 &          4 213 597 \\
     7 &     113 634 &        190 899 322 \\
     8 &     853 467 &     10 480 142 147 \\
     9 &   6 536 382 &    682 076 806 159 \\
    10 &  50 852 019 & 51 724 158 235 372 \\
    \hline
  \end{tabular}
  \renewcommand\arraystretch{0.7}
  \caption{Sizes of $\Mot_n$ and $\Prt_n$ for small values of $n$}
  \label{tab:mn-size}
\end{table}

\section{Preliminaries}
\label{sec:motzkin-prelim}


\section{Congruence lattice of $\Mot_n$}
\label{sec:motzkin-congs}


\section{Other monoids}
\label{sec:motzkin-other}
