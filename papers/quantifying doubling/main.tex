\documentclass[12pt]{amsart}
\usepackage{amsmath, amsfonts, amsthm, amssymb}
\usepackage{graphicx}
\usepackage{float}
\usepackage{color}
\usepackage{tikz}
\usepackage{xcolor}
\usepackage{caption}
\usepackage{subcaption}
\usepackage{natbib}
\usepackage{verbatim}
%\usepackage{pdfsync}
%\usepackage{showkeys}






\hoffset=-1.8cm\voffset=-1.1cm
\setlength{\textwidth}{16.5cm}
\setlength{\textheight}{22.6cm}



\setcounter{secnumdepth}{3}
\numberwithin{equation}{section}


\sloppy

\newtheorem{thm}{Theorem}[section]
\newtheorem{lma}[thm]{Lemma}
\newtheorem{cor}[thm]{Corollary}
\newtheorem{defn}[thm]{Definition}
\newtheorem{cond}[thm]{Condition}
\newtheorem{prop}[thm]{Proposition}
\newtheorem{conj}[thm]{Conjecture}
\newtheorem{rem}[thm]{Remark}
\newtheorem{ques}[thm]{Question}
\newtheorem{property}[thm]{Property}
\newtheorem{exam}[thm]{Example}

\renewcommand{\ge}{\geqslant}
\renewcommand{\le}{\leqslant}
\renewcommand{\geq}{\geqslant}
\renewcommand{\leq}{\leqslant}
\renewcommand{\H}{\text{H}}
\renewcommand{\P}{\text{P}}
\renewcommand{\l}{\overline{\dim}_{\text{loc}}(x,\mu)}
\newcommand{\urdim}{\overline{\dim}_{\textup{reg}}}
\newcommand{\lrdim}{\underline{\dim}_{\textup{reg}}}
\renewcommand{\a}{\dim_{\textup{A}}}
\renewcommand{\epsilon}{\varepsilon}

\newcommand{\Height}{8}
\newcommand{\Width}{8}

\allowdisplaybreaks







\begin{document}

\title[quantifying doubling]{Regularity dimensions: quantifying doubling and uniform perfectness}



\author[D. C. Howroyd]{Douglas C. Howroyd}
\address{Douglas C. Howroyd\\
School of Mathematics \& Statistics\\University of St Andrews\\ St Andrews\\ KY16 9SS\\ UK  }
\curraddr{}
\email{dch8@st-andrews.ac.uk}



\subjclass[2010]{ primary: 28A80; secondary: 37C45, 28C15, 60G18, 60G51, 60G52, 60J65.}

\keywords{upper regularity dimension, lower regularity dimension, Assouad dimension, lower dimension, doubling measure, uniformly perfect measure, quasisymmetric homeomorphisms, Brownian motion, Levy process.}




\begin{abstract}
We use the \emph{upper and lower regularity dimensions} to quantify doubling and uniformly perfect measures respectively. Relations between these two regularity properties are discussed and applications to Diophantine approximation are introduced. The upper regularity dimension of pushforwards of doubling measures onto graphs of Brownian motion is also calculated.
\end{abstract}


\maketitle




\section{Introduction} \label{intro}\label{intro}

\textcolor{red}{just need to add pics, do abstract and add refs}

When studying metric spaces and measures on them some basic regularity properties are often assumed. This allows us to avoid many of the common pathological counterexamples. A trivial regularity property could be to assume that a set is non-empty. However, over time, more sophisticated characteristics have been introduced and studied. We will be interested in studying two regularity properties of measures, \textit{doubling} and \textit{uniform perfectness}, and how they interact with concepts of dimension, analogous purely metric regularity properties and each other.

Both spaces and measures can be doubling or uniformly perfect. A metric space $X$ is doubling if there is a constant $C > 0$ such that $B(x,R)$ can be covered by at most $C$ balls of radius $R/2$ for all $x\in X$ and $R > 0$. Unless stated otherwise $B(x,R)$ is the open ball of centre $x$ and radius $R$. Such spaces are particularly well behaved, for instance Assouad's embedding theorem says doubling spaces can be embedded into Euclidean space in a nearly bi-Lipschitz way, see \cite{Assouad}. The \textit{Assouad dimension} can be thought of quantifying how doubling a space is, since a space is doubling if and only if it has finite Assouad dimension. This notion of dimension has also seen much dimension theoretic interest, interacting in a number of ways with the some of the more classical notions of dimension. For an introduction to the Assouad dimension, a formal definition and some recent results see \cite{probs jons stoof}. Of particular interest will be the calculation of the Assouad dimension of graphs of Brownian motion in \cite{me and han}.

In a similar way, a measure $\mu$ on a metric space $X$ is said to be \textit{doubling} if there is a constant $C(2) > 1$ such that $\mu(B(x,R)) \le C(2) \mu(B(x,R/2))$ for all $x \in X$ and $R> 0$. Note that if a space is doubling, $2$ can be replaced by any $\theta > 1$ in this definition, obtaining new doubling constants $C(\theta)$. That is 
\[
\mu(B(x,R)) \le C(\theta) \mu(B(x,R/\theta))
\]
for $\theta > 1$, $x\in X$ and $R > 0$. A natural analogue of the Assouad dimension for measures comes from \cite{lukainen?} and was first studied in \cite{Anti} under the name of \textit{upper regularity dimension}. The upper regularity dimension of $\mu$ is defined by 
\begin{multline*} 
\overline{\dim}_{\text{reg}} \mu = \inf \Bigg\{ s \geq 0 \, \,  : \,  \text{ there exists a  constant }C  > 0\text{  such that, for all  $0< r< R $} \\ \text{  and all $x \in \text{supp} (\mu)$, we have }  \ \  \frac{\mu(B(x,R))}{\mu(B(x,r))} \leq C\left(\frac{R}{r}\right)^{s} \Bigg\},
\end{multline*}
taking $\inf \emptyset = + \infty$. Like the Assouad dimension, the upper regularity dimension quantifies how doubling a measure is since a measure has finite upper regularity dimension if and only if it is doubling. This notion of dimension has mostly been studied in the traditional fractal geometry setting, as in \cite{peeps}. We will expand on some of the basic properties of this dimension.



Many of the same ideas apply to the concepts of uniformly perfect spaces and measures. A space $X$ is uniformly perfect if there exists a constant $K>1$ such that for any $x,r$ there exists $y \in X \cap( B(x,r) \setminus B(x,r/K))$. Uniformly perfect sets have been studied for some time, especially in the setting of fractal geometry, for instance \cite{need refs here}. One can in fact show that a set is uniformly perfect if and only if it has positive lower dimension, the lower dimension being a natural analogue of the Assouad dimension. Many of the previous references for the Assouad dimension discuss the lower dimension if further details are desired.

Following the common thread, one can state a definition for a measure to be \textit{uniformly perfect}; such measures were also recently called reverse-doubling due to the definition resembling that of doubling measures. Given a measure $\mu$ on a space $X$, $\mu$ is uniformly perfect if there exists a constant of uniform perfectness $K(2)$
\[
\mu(B(x,R)) \ge K(2) \mu(B(x,R/2))
\]
for all $x\in X$ and $R > 0$. As before we can replace $2$ by $\theta > 1$ in this definition. 

Similarly the \textit{lower regularity dimension} of a measure $\mu$ is defined by 
\begin{multline*} 
\underline{\dim}_{\text{reg}} \mu = \sup \Bigg\{ s \geq 0 \, \,  : \,  \text{ there exists a  constant }C  > 0\text{  such that, for all  $0< r< R $} \\ \text{  and all $x \in \text{supp} (\mu)$, we have }  \ \  \frac{\mu(B(x,R))}{\mu(B(x,r))} \geq C\left(\frac{R}{r}\right)^{s} \Bigg\}.
\end{multline*}
The lower regularity dimension interacts with the notion of uniform perfectness in much the same way that the upper regularity dimension and the doubling property work together. That is, a measure has positive lower regularity dimension if and only if it is uniformly perfect. We will explore some properties of the lower regularity dimension and study the links with other notions of regularity, including the upper regularity dimension. Many of the references discussing the upper regularity dimension also consider the lower regularity dimension.

Whilst we will not need geometric definitions of the Assouad and lower dimensions, for clarity, one could think of them in terms of the regularity dimensions in the following way
\[
\dim_{\text{A}} F = \inf \left\{ \overline{\dim}_{\text{reg}} \mu \,  \colon \, \mu \text{ is a Borel probability measure supported on } F\right\}.
\]
\[
\dim_{\text{L}} F = \sup \left\{ \underline{\dim}_{\text{reg}} \mu \,  \colon \, \mu \text{ is a Borel probability measure supported on } F\right\}.
\]




\section{Results}

\subsection{Quantifying an example of Heinonen}


When studying doubling measures a technical proposition is often employed to truly benefit from the regularity of these measures. Simply put, a doubling measure on a uniformly perfect space is also a uniformly perfect measure. This implies the following important bounds. Say $\mu$ is such a measure on a bounded space $X$, then there exists constants $0< \lambda_1, \lambda_2 < \infty$ and $0 < t \le s < \infty$ such that for any $x \in X$ and $0 < r$
\[
\lambda_1 r^s \le \mu(B(X,r)) \le \lambda_2 r^t.
\]
It is not clear where this was first stated, but a standard reference \ref{heinon} provides this result as an example (\ref{}num whatever) without a proof. 

We wish to quantitatively improve this result using the regularity dimensions. More precisely, given a measure $\mu$ of fixed upper regularity dimension $s$ on a space $X$ of fixed lower dimension $l$, can we bound the lower regularity dimension of $\mu$ as a function of $s$ and $l$? The following result does not quite answer this question as it returns a function of the doubling and uniform perfectness constants. However, as we will see afterwards, this is closer to the desired solution than it appears.

\begin{prop}
If $X$ is uniformly perfect and $\mu$ is doubling then $\mu$ is uniformly perfect. In particular if $X$ is $K$-uniformly perfect and $\mu$ doubling with doubling constants $C(\theta)$ then  
$$\lrdim \mu \ge \frac{\ln(1-C^{-1}(8K))}{1-\ln(4K)} .$$  
\end{prop}



It would have been preferable to obtain a lower bound depending on the lower dimension of $X$ and the upper regularity dimension of $\mu$. Clearly the lower regularity dimension of $\mu$ must depend on the lower dimension of $X$, since the lower dimension is an upper bound, see \ref{me and jon?}. However, as we did not obtain a result depending on the upper regularity dimension, we are lead to ask if there exists a uniformly perfect space and a sequence of doubling measures on that space which all have the same upper regularity dimension but whose lower regularity dimensions can be made as small as possible. A cursory check of some standard examples such as self-similar sets and measures implies this might not be feasible. 

The question then becomes whether the upper regularity dimension is even distinct from the doubling constants. In \ref{fraser-howroyd}, the upper regularity dimension was shown to be bounded  above by a function of the doubling constants. The following shows that this bound is actually an equality and thus the upper regularity dimension depends explicitly on the constants. Combining this result with the previous proposition, it follows that a doubling measure on a uniformly perfect space must have lower regularity dimension bounded below by a function of the upper regularity dimension and the constant of uniform perfectness.

\begin{thm}
Let $\mu$ be a doubling or uniformly perfect measure with doubling constants $C(\theta)$ or constants of uniform perfectness $K(\theta)$ respectively. When $\mu$ is doubling, the upper regularity dimension is exactly equal to $\inf_{\theta > 1}\frac{\log C(\theta)}{\log \theta}$. Similarly, when $\mu$ is uniformly perfect, the lower regularity dimension equals $\sup_{\theta > 1} \frac{\log K(\theta)}{\log \theta}$.
\end{thm}


We formally state the answer to our original question now, as a direct corollary of the previous results.

\begin{cor}
Let $X$ be a metric space of positive lower dimension and $\mu$ be a measure which is doubling, then $\mu$ is uniformly perfect quantitatively, depending on the constant of uniform perfectness of $X$ and the upper regularity dimension of $\mu$.
\end{cor}

\begin{proof}
This follows from the previous results.
\end{proof}


We finish this section with a brief discussion of the sharpness of this corollary. Assuming positive lower dimension of the space is required here as it is simple to construct spaces of zero lower dimension but finite Assouad dimension. In such a setting there must exist a measure of upper regularity dimension close to the Assouad dimension of the space and so doubling. But any measure on this space can not be uniformly perfect as the lower regularity dimension is a lower bound to the lower dimension. A trivial such example would be the set of points $\left\{ 1/n : n \in \mathbb{N} \right\}$ with the Euclidan metric. This set is know to have zero lower dimension but Assouad dimension 1; doubling measures on this space were constructed in \ref{fraser-howroyd}.

There are many examples of uniformly perfect measures, even on doubling spaces, that are not doubling, so we cannot interchange the two notions and obtain an analogous result.  For instance, one can take a self-similar set with overlaps, this is a doubling space. There are numerous uniformly perfect measures on such a space that are not doubling. Thus a uniformly perfect measure on a doubling and uniformly perfect space need not be doubling.







\subsection{Regularity dimensions under quasisymmetric homeomorphisms}


Quasisymmetric homeomorphisms are a generalisation of bi-Lipschitz maps, preserving relative sizes but not necessarily global size which were first introduced in \ref{behring?-ahlfors and tukia-vaisala}. In the Euclidean setting quasisymmetric homeomorphisms are equivalent to the often studied quasiconformal homeomorphisms. A homeomorphism $f\colon X \rightarrow Y$ is an \textit{$\eta$-quasisymmetric homeomorphism} if there is a homeomorphism $\eta \colon [0,\infty) \rightarrow [0,\infty)$ such that 
\[
\lvert x - a \rvert \le t\lvert x - b \rvert
\]
implies 
\[
\lvert f(x) - f(a) \rvert \le \eta(t) \lvert f(x) - f(b) \rvert
\]
for all $x,a,b \in X$ and for all $t>0$.


Equivalently there exists a homeomorphism $\eta$ as above such that 
\[
\frac{d_Y(f(x),f(y))}{d_Y(f(x),f(z))} \le \eta \left(\frac{d_X(x,y)}{d_X(x,z)} \right)
\]
for any distinct points $x,y,z \in X$. Here $\eta$ is not unique for a given quasisymmetric homeopmorphism. In fact if such an $\eta$ exists, then any homeomorphism $\eta'$ which is always greater than $\eta$ also works. 


A property of particular interest to us is that doubling and uniform perfectness of spaces are quasisymmetric invariants. This can be quantified, so there are bounds on the Assouad and lower dimensions of images of spaces under quasisymmetric embeddings. We wish to know if the same holds for doubling and uniformly perfect measures. In particular we will study \textit{pushforward measures} under quasisymmetric homeomorphisms. Given a measure $\mu$ on a space $X$ and $f$ a map from $X$ to some space $Y$, the pushforward measure of $\mu$ under $f$ is denoted $f_*\mu$ and is defined by
\[
f_*\mu (A) = \mu(f^{-1}(A))
\]
for any measureable subset $A$ of $Y$, where $f^{-1}(A) = \left\{x \in X \colon f(x) \in A \right\}$. 

To avoid having trivial upper and lower regularity dimensions of $\mu$ it is reasonable to assume that $X$ is doubling and uniformly perfect. This then lets us employ the following theorem. 

\begin{thm}[11.3 \ref{Heinonen}]
A quasisymmetric homeomorphism $f$ of a uniformly perfect space $X$ is $\eta$-quasisymmetric with $\eta$ of the form
\[
\eta(t) = E \max\left\{t^\alpha, t^{1/\alpha}\right\},
\]
where $E\ge 1 $ and $\alpha \in (0,1]$ depend only on $f$ and $X$.
\end{thm}

For clarity we will often write $\eta_{\alpha}$ to indicate the homeomorphism $\eta$ associated with the constant $\alpha$ as described here. Recall that such an $\eta$ need not be unique. In fact if one can find an $\alpha$ as in the theorem, then the statement also holds for all $0< \alpha' \le \alpha$. We therefore assume that the chosen $\alpha$ is as large as possible to optimise the following bounds.

\begin{thm}
Let $X$ be a uniformly perfect space and $\mu$ be doubling on $X$. When $f$ is an $\eta_\alpha$-quasisymmetric homeomorphism the following bounds hold
\[
\alpha \, \urdim \mu \le \urdim f_{*}\mu \le \urdim \mu/\alpha
\]
and 
\[
\alpha \, \lrdim \mu \le \lrdim f_{*}\mu \le \lrdim \mu/\alpha
\]
where $f_{*}\mu = \mu \circ f^{-1}$ is the pushforward of $\mu$.
\end{thm}




\subsection{Pushforwards of measures onto graphs of Brownian motion}


Having shown that the regularity dimensions are well behaved under quasisymmetric maps we now turn our attention to random maps and ask if doubling and uniform perfectness are also preserved in these situations. Specifically we will consider maps from the unit interval onto graphs of L\'evy processes.

A \textit{L\'evy process} is a random function $X : [0,\infty) \rightarrow \mathbb{R}$ satisfying:
\begin{enumerate}
    \item with probability 1, $X(0) = 0$;
    
    \item $X(t)$ is right continuous and has left limits at every point $t$;
    
    \item $X(t+h) - X(t)$ is equal to $X(h)$ in distribution for all $t,h >0$ (stationary increments);
    
    \item for all $0<t_1 < t_2 \cdots < t_k$, the increments $X(t_i) - X(t_{i-1})$ are independent;
    
\end{enumerate}

When the distribution of increments is chosen to be the Normal distribution with mean 0 and variance $h$ we recover the Wiener process (or Brownian motion). L\'evy processes are fundamental tools in several areas of mathematics and have been extensively studied. Their fractal properties were first investigated by \cite{taylor 53} where the Hausdorff dimension of the graph of Brownian motion was shown to be almost surely equal to $3/2$ and the range of $d$-dimensional Brownian motion was found to have dimension $2$ almost surely for any $d \ge 2$. Since then further work has calculated other dimension theoretic properties of these objects such as in \cite{box dim, ass dim etc?}. 

Whilst many geometric objects associated to L\'evy processes have been studied, we will only consider properties of the graph of L\'evy processes. Given a L\'evy process $X$, the graph of $X$ above the unit interval is defined by:
\[
G_X^{[0,1]} = \left\{ (t,X(t)) \colon t \in [0,1] \right\}.
\]
More generally, denote by $G_X^I$ the graph of the process $X$ above the interval $I \subseteq [0,1]$. There is a natural, associated map $f: [0,1] \rightarrow \mathbb{R}^2$ which maps the unit interval to the graph of the process. This will be the map we wish to use to construct pushforward measures and for the rest of this section, $f$ should be assumed to be this map. 

One of the interesting features of L\'evy processes is their statistical self-affinity. Not all processes have this property so we will restrict to \textit{stable} or $\alpha$-\textit{stable processes}, that is for some $\alpha > 0$
\[
a^{-1/\alpha}X(at) =^d X(t)
\]
for all $a,t > 0$ and $=^d$ means equal in distribution. For example the Wiener process is 2-stable. In fact all stable processes have $\alpha \in (0,2]$. 

Our final condition is a simple assumption that $X(1)$ is non-vanishing on $\mathbb{R}$. Non-zero on an interval would also work, this is just to ensure the graphs are not just flat lines.

This leads us to the question of this section: given a doubling measure $\mu$ on the unit interval, is $f_*\mu$ also doubling? A similar question holds for uniformly perfect measures. In \cite{han and I}, it was shown that the Assouad dimension of $G_X^{[0,1]}$ is almost surely 2 so there must exist at least one doubling measure on the graph. However, this might be a highly irregular measure and not be representative of the global behaviour. For the Hausdorff dimension, the proof by Taylor \textcolor{red}{check this} shows that the Hausdorff dimension of the pushforward of Lebesgue measure almost surely attains the dimension of the graph itself. It turns out that this is usually not the case for the regularity dimensions.

\begin{thm}
Let $\mu$ be a doubling measure on $[0,1]$ and $X$ a stable L\'evy process with $X(1)$ being non vanishing on $\mathbb{R}$. Then $f_*\mu$ is almost surely not doubling on $G_X^{[0,1]}$. Also, $f_*\mu$ is almost surely not uniformly perfect.


\end{thm}


Trivially this implies the upper regularity dimension of $f_*\mu$ is almost surely infinity. Therefore any measure whose upper regularity dimension approximates the dimension of the graph is highly dependent on the specific graph and so there is no one measure that attains the dimension for most realisations, unlike the Hausdorff case. 




\subsection{Uniformly perfect and $\alpha$-decaying measures}

\textcolor{red}{do this section}
It is known that the lower regularity dimension of a measure is positive if and only if the measure is uniformly perfect, see \cite{antti or fraser-howroyd}. A property that has appeared recently in Diophantine approximation is the notion of weakly absolutely $\alpha$-decaying. This was first introduced in \cite{Sanju et al} following the previous uses of friendly measures by \cite{Kleinvock, Lindenstrauss and Weiss} and quasi-decaying measures by \cite{Das, Fishman, Simmons, Urbanski}. A measure $\mu$ is weakly absolutely $\alpha$-decaying for some $\alpha > 0$ when there exists constants $C, R_0 >0$ such that for all $\varepsilon > 0$
\[
\mu(B(x,\varepsilon R)) \le C \varepsilon^{\alpha} \mu(B(x,R))
\]
for all $x \in X$ and $R<R_0$.

This property has some resemblance to the ideas of doubling and uniformly perfect measures. We show the following.
\begin{prop}
A measure $\mu$ has lower regularity dimension equal to $\alpha > 0$ if and only if it is weakly absolutely $\alpha$-decaying. 
\end{prop}


This result actually leads to an equivalent but hopefully more detailed statement of Theorem 2 in \cite{sanju et al}. Their theorem concerns the size of the set of points which approximate a given point $y$ well with respect to some function $\psi$ for certain subsets of Kleinian manifolds. In the Euclidean setting, one usually considers the Lebesgue measure of this set. However when working on different manifolds other measures must be considered that somewhat mirror the regularity of Lebesgue measure. This leads to the notion of $\alpha$-decaying. Without further elaborating on the definitions we state both the old and new versions of the theorem.

\begin{thm}[sanju et al]
Let $G$ be a nonelementary, geometrically finite Kleinian group and let $y$ be a parabolic fixed point of $G$, if there are any, and a hyperbolic fixed point otherwise. Fix $\alpha > 0$, and let $K$ be a compact subset of $\Gamma$ equipped with a weakly absolutely $\alpha$-decaying meausre $\mu$. Then
\[
\mu(K\cap W_y(\psi)) = 0 \quad if \quad \sum_{r=1}^\infty r^{\alpha-1} \psi(r)^{\alpha} < \infty.
\]
\end{thm}

\begin{thm}
Assume $G$ and $y$ are as above. Let $K$ be a compact subset of $\Gamma$ with lower dimension equal to $s > 0$. For any $0 < \alpha < s$, if there exists a weakly absolutely $\alpha$-decaying measure $\mu$ on $K$, then
\[
\mu(K\cap W_y(\psi)) = 0 \quad if \quad \sum_{r=1}^\infty r^{\alpha-1} \psi(r)^{\alpha} < \infty.
\]
In particular, if  $\sum_{r=1}^\infty r^{s-1} \psi(r)^{s} < \infty$ then any weakly absolutely $\alpha$-decaying measure on $K$ is such that $\mu(K\cap W_y(\psi)) = 0$. One can also find a sequence of weakly absolutely $\alpha_n$-decaying measures on $K$ such that $\alpha_n \rightarrow s$. 
\end{thm}

The proof of Theorem 2.9 is the same as the proof in \cite{sanju et al}, simply noting that positive lower dimension implies the existence of measures. As such we do not reproduce the proof here, for the interested reader the original proof is very accessible.

Weakly absolutely decaying measures were the correct measures to consider in the setting of Kleinian manifolds whereas friendly measures were used in the context of subsets of Euclidean space. It would be a natural extension to study the links between friendly measures and the regularity dimensions, especially as one of the conditions for a measure to be friendly is that it is doubling.  


\section{Proofs}

This section will be broken into several subsections that are mostly independent of each other but the notation will remain consistent throughout. In section 3.1 we cover the results found in 2.1. In section 3.2 we prove Theorem 2.5. Section 3.3 is dedicated to measures on graphs of L\'evy processes. Finally in section 3.4 a short proof of Proposition 2.8 is provided.



\subsection{Quantifying an example of Heinonen}



\begin{proof}[Proof of Proposition 2.1]
Let $X$ and $\mu$ be as in the statement of the proposition. We will rework the proof found in \ref{}(Anti DIMENSIONS, WHITNEY COVERS, AND TUBULAR NEIGHBORHOODS)[lemma 3.1], paying careful attention to the constants in play. 

To start, a technical result is required. Proposition B.4.6, in \ref{Gromov}, states that in our setting there exists a constant $a \in (0,1)$ such that 
\[
\mu(B(x,aR)) \le (1-a) \mu(B(x,R))
\]
for any $x,R$. We start by determining $a$ as a function of our known constants. 

For any $x\in X$ and $R>0$, as $X$ is uniformly perfect, there exists a $y \in X$ such that $$\frac{R}{2K} \le d(x,y) \le \frac{R}{2}.$$ This choice of $y$ ensures that $B(x,\frac{R}{4K}) \cap B(y,\frac{R}{4K}) = \emptyset $ and $B(x,\frac{R}{4K}) \cup B(y,\frac{R}{4K}) \subseteq B(x,R)$. Thus
\begin{align*}
\mu(B(x,\frac{R}{4K})) &\le \mu(B(x,R)) - \mu(B(y, \frac{R}{4K}))\\
& \le \mu(B(x,R)) - C(8K)^{-1}\mu(B(x,R)) \\
& = (1-C(8K)^{-1}) \mu(B(x,R)).
\end{align*}
By iterating this construction we obtain
\[
\mu(B(x,R/(4K)^n)) \le (1-C(8K)^{-1})^n \mu(B(x,R))
\]
for any $n\in \mathbb{N}$, as desired.

Returning to the actual question, fix $x\in X$, $0 < r < R$ and choose $n\in \mathbb{N}$ such that $(4K)^{-n-1}R < r \le (4K)^{-n}R$ so that $B(x,r) \subseteq B(x,R/(4K)^{n})$. Then
\begin{align*}
\frac{\mu(B(x,R))}{\mu(B(x,r))} \ge& \frac{\mu(B(x,R))}{(1-C(8K)^{-1})^n\mu(B(x,R))} \\
& \ge (1-C(8K)^{-1})^{\frac{-\ln(R/r)}{\ln(4K)-1}}\\
& = \left(\frac{R}{r}\right)^{\frac{\ln(1-C(8K)^{-1})}{1-\ln(4K)}}
\end{align*}
as desired.

\end{proof}

Note that in the proof of \ref{}[Gromov] one should use the optimal doubling  and uniform perfectness constants to obtain the best bound possible, however the result itself is likely not sharp.
 
 Now we turn our attention to the relationship between the doubling constants and the upper regularity dimension, as well as the constants of uniform perfectness and the lower regularity dimension.

\begin{proof}[Proof of Theorem 2.2]
We start by proving the link between the upper regularity dimension and the doubling constants. The upper bound follows from \ref{}[on the upper regularity dimensions of measures], the difference in formula is purely notational. To obtain a lower bound on the upper regularity dimension of a measure $\mu$ on a space $X$, it suffices to find, for any $s\le \inf\frac{\log C(\theta)}{\log \theta}$, a sequence of $x\in X$ and $0<r<R$, with $R/r \rightarrow \infty$, such that 
\[
\frac{\mu(B(x,R))}{\mu(B(x,r))} \ge \left(\frac{R}{r}\right)^s.
\]


From the definition of doubling we know that $\mu(B(x,\theta r) ) \le C(\theta) \mu(B(x,r))$ for any suitable $x,r, \theta$. Fixing $\theta$ we pick $C(\theta)$ to be sharp in the sense that for any $\varepsilon>0$ there exists at least one pair of $x,r$ such that $\mu(B(x,\theta r) ) \ge (C(\theta)-\varepsilon) \mu(B(x,r))$. 

Let $0 < \varepsilon< \frac{1}{2}$ and $s \le \inf_{\theta > 1}\frac{\log C(\theta)}{\log \theta}$. To choose our sequence of $x,r$ and $R$ we simply pick any sequence of increasing $\theta$. Then from our choice of $C(\theta)$ with respect to the fixed $\varepsilon$, the pair $x$ and $r$ are the pair obtained above. $R$ is then fixed by $R = \theta r$. Then, due to the choice of $s$,
\[
\frac{\mu(B(x,R))}{\mu(B(x,r))} = C(\theta)-\varepsilon \ge \frac{1}{2} C(\theta)  \ge \frac{1}{2}\theta^s = \frac{1}{2}\left(\frac{R}{r} \right)^s,
\]
completing the proof.

The lower regularity dimension result follows similarly.
\end{proof}









\subsection{Regularity dimensions under quasisymmetric homeomorphism}



Whilst Theorem \ref{heinonenalpha} of \ref{heinonen} is the key ingredient in the proof of our theorem, the following proposition which can also be found in \cite{heinonen} is also required.

\begin{prop}[10.6 \cite{Heinonen}]
When a quasisymmetric homeomorphism $f\colon X \rightarrow Y$ is $\eta$-quasisymmetric, its inverse $f^{-1}$ is an $\eta'$-quasisymmetric homeomorphism of the form $\eta'(t) = 1/\eta^{-1}(1/t)$ for $t>0$.
\end{prop}

It is then clear that a quasisymmetric homeomorphism $f$ on a uniformly perfect space is associated with a homeomorphism $\eta = E\max\left\{t^\alpha, t^{1/\alpha}\right\}$ and $f^{-1}$ is also a quasisymmetric homeomorphism associated with the function $1/\eta^{-1}(1/t) = E \min\left\{t^\alpha, t^{1/\alpha} \right\}$.


\begin{proof}[Proof of Theorem 2.5]
We start by proving the upper bound for the upper regularity dimension. Let $y\in Y$ and $0<r<R$. Since $Y$ is uniformly perfect, we can find $z_1,z_2$ such that $z_1\in B_Y(y,2R) \setminus B_Y(y,R)$ and $z_2 \in B_Y(y,2r) \setminus B_Y(y,r)$. 


Choose any point $a \in B_Y(y,R)$. From our choice of $z_1$, it is clear that $\lvert y - a \rvert \le \lvert y - z_1 \rvert$. Thus, as $f$ is quasisymmetric $\lvert f^{-1}(y) - f^{-1}(a) \rvert \le \eta(1) \lvert f^{-1}(y) - f^{-1}(z_1) \rvert$, and so 
\[
f^{-1}(B_Y(y,R) \subseteq B_X(f^{-1}(y),\eta(1)d_X(f^{-1}(y), f^{-1}(z_1)).
\]
Similarly, choosing $a \in B_Y(y,R) \setminus B_Y(y,R/2)$, we have $\lvert y - a \rvert \ge \lvert y - z_1 \rvert/4$ and so $$\lvert f^{-1}(y) - f^{-1}(a) \rvert \ge \eta^{-1}(4)\lvert f^{-1}(y) - f^{-1}(z_1) \rvert.$$
Hence 
\[
f^{-1}(B_Y(y,R)) \supseteq B_X(f^{-1}(y),\eta^{-1}(4)d_X(f^{-1}(y),f^{-1}(z_1))).
\]



Similar statements clearly hold for $r$ with $z_2$. Thus
\begin{align*}
    \frac{\mu(f^{-1}(B_Y(y,R)))}{\mu(f^{-1}(B_Y(y,r)))} &\le \frac{\mu(B_X(f^{-1}(y),\eta(1)d_X(f^{-1}(y), f^{-1}(z_1)) ))}{\mu(B_X(f^{-1}(y),\eta^{-1}(4)d_X(f^{-1}(y),f^{-1}(z_2)) ))} \\
    & \le C_{\urdim \mu}\left( \frac{\eta(1)d_X(f^{-1}(y),f^{-1}(z_1))}{\eta^{-1}(4)d_X(f^{-1}(y),f^{-1}(z_2))}\right)^{\urdim \mu} \\
    & \le C_{\urdim \mu}E \eta(1)^{\urdim \mu}/\eta^{-1}(4)^{\urdim \mu} \left(\frac{d_Y(y,z_1)}{d_Y(y,z_2)} \right)^{\urdim \mu / \alpha} \\
    & \le C_{\urdim \mu}E \eta(1)^{\urdim \mu}/\eta^{-1}(4)^{\urdim \mu} \left(\frac{2R}{r} \right)^{\urdim \mu / \alpha}
\end{align*}
as desired, where $C_{\urdim \mu}$ is the constant from the definition of the upper regularity dimension of $\mu$.

For the lower bound of the upper regularity dimension, first note that for any $\varepsilon > 0$ there exists a sequence of $x_k \in X$ and $0< r_K < R_k$, with $R_k / r_k$ tending to infinity, such that for all $k\in \mathbb{N}$
\[
\frac{\mu(B(x_k,R_k))}{\mu(B(x_k,r_k))} \ge C_{\urdim \mu} \left(\frac{R_k}{r_k} \right)^{\urdim \mu-\varepsilon}.
\]
We will fix $\varepsilon>0$ and the corresponding sequences for the rest of this proof.

Whilst the upper regularity dimension of the pushforward will not necessarily be obtained by considering the image of these sequences, it will allow us to find a lower bound. The difficulty lies in constructing suitable images of these sequences. The natural image of the sequence $(x_k)$ is just $(f(x_k) )$. To find the corresponding diameters we fix $x\in X$ and $R>0$. Then define $R^{\sup} = \sup_{y \in B(x,R)} \lvert f(x) - f(y) \rvert$ and $R^{\inf} = \inf_{y \in B(x,R) \setminus B(x,R/2)} \eta(1) \lvert f(x) - f(y) \rvert$, note $R_{\inf}$ is not quite the first number with the supremum replaced by the infimum. $z_{R}^{\sup} \in X$ is chosen to be the point for which the supremum is attained and $z_{R}^{\inf}$ comes from the infimum. The supremum and infimum are attained as $B(x,R)$ is compact and $f(x)$ is a quasisymmetric homeomorphism. We can also see that $z_{R}^{\sup} \neq x \neq z_{R}^{\inf}$. If $R_k/r_k \rightarrow \infty$ as $k$ grows, then so will $R_k^{\sup}/r_k^{\inf}$. Thus we have for any $k\in \mathbb{N}$
\begin{align*}
\frac{f_* \mu(B(f(x_k),R_k^{\sup}))}{f_*\mu(B(f(x_k),r_k^{\inf}))} &\ge \frac{\mu(B(x_k,R_k))}{\mu(B(x_k,r_k))} \\
& \ge  C_{\urdim \mu} \left(\frac{R_k}{r_k} \right)^{\urdim \mu-\varepsilon} \\ 
& \ge C_{\urdim \mu} \left ( \frac{\lvert x- z_{R_k}^{\sup} \rvert}{2\lvert x- z_{r_k}^{\inf} \rvert}\right)^{\urdim \mu-\varepsilon} \\
& \ge \frac{C_{\urdim \mu}E}{2^{\urdim \mu-\varepsilon}} \left(\frac{\lvert f(x) - f(z_{R_k}^{\sup})\rvert}{\lvert f(x)- f(z_{r_k}^{\inf}) \rvert} \right)^{\alpha(\urdim \mu-\varepsilon)} \\
& = \frac{C_{\urdim \mu}E}{2^{\urdim \mu-\varepsilon}}\left(\frac{R_k^{\sup}}{r_k^{\inf}} \right)^{\alpha(\urdim \mu-\varepsilon)}
\end{align*}
completing the proof.

Proofs for the lower regularity dimension follow similarly.

\end{proof}









\subsection{Pushforward of measures onto graphs of Brownian motion}


Choose a L\'evy process $X$ which satisfies the conditions in Theorem \ref{brownianThm} with scaling coefficient $\alpha$ and fix a graph $G_X^{[0,1]}$ realised by this process. Start by assuming $\alpha > 1$, the proof will work in the same way for $\alpha < 1$ given a slight modification which will be commented on later in the proof. $\mu$ is taken to be a doubling measure on the unit interval. Recall $f$ is defined to be the function which maps the unit interval to the graph of our L\'evy process and $f_*\mu$ is the pushforward measure of $\mu$ onto the graph that we wish to study. 

We start by calculating the upper regularity dimension of $f_*\mu$. Let $s>0$. The general strategy for this proof is to find a sequence of events that are all independent and have positive probability. Then a simple application of the Borel-Cantelli lemma will yield that almost surely these events will happen infinitely often. By choosing our events carefully this will yield a sequence of balls that show the upper regularity dimension of the pushforward measure must be greater than $s$. As $s$ is abitrary, this will conclude the proof.

Given our $\alpha$-scaling L\'evy process, we define the rectangle centered at $a\in [0,1]$ with side lengths $R_1,R_2$ by $Rec(a,R_1,R_2) = I(a,R_1) \times I(X(a),R_2)$ where $I(a,R) = [a-R/2,a+R/2]$ is just an interval of length $R$ and centre $a$. $G_X^{I(x,R)}$ will denote the graph of $X$ above the interval $I(x,R)$.

The particular events $E_i$ we are interested in are defined as follows: let $x_i \in [0,1]$, $R_i > r_i> 0$ and $\beta > 1$, then $E_i$ is the event in which $G_X^{I(x_i,R_i^{\alpha})} \subset Rec(x_i,R_i^{\alpha},R_i)$ and $Rec(x_i, r_i, r_i^{1/\alpha}) \cap G_X^{I(x_i,r_i)} = G_X^{I(x_i,r_i^{\beta})}$. These events are chosen so that the measure on the graph will be `large' on the rectangle of small side length $R^{\alpha}$ but `small' on the rectangle of small side length $r$. Figure \ref{brownian_event} is a geometric representation of such an event.  

Note that $\alpha>1$ here is important for the rectangles to be tall and thin. If $\alpha < 1$ it would suffice to change $Rec(x_i,R_i^{\alpha}, R_i)$ to $Rec(x_i, R_i, R^{\alpha})$ and similarly for the smaller rectangles. The rest of the proof would run in the same way afterwards with some slight changes in the calculations of $\beta$ at the end.

Given any sequences $x_i \in [0,1]$, $R_i > r_i > 0$ and $\beta > 1$ we can consider the associated events $E_i$ as above. To make sure the `smaller' rectangle is actually smaller, assume $R_i^{\alpha} > r_i$ without loss of generality. If $Rec(x_m,R_m^{\alpha}, R_m) \cap Rec(x_n,R_n^{\alpha},R_n) = \emptyset$ for all $m\neq n$, then the events are all independent due to the independent increment property of the L\'evy process. As long as the distribution of $X(1)$ is non-vanishing on the unit interval, the probability of any of these events is positive.

We can now choose our sequence of events. Start by picking any disjoint sequence of reals  $x_i$, $\left\{1-2^{-i} \right\}_{\mathbb{N}}$ would suffice. Then the $R_i$ are taken so that the intervals $I(x_i, R_i)$ do not overlap ensuring independence, say $4^{-i}$. Initially any sequence of $r_i$ can be chosen as long as $R_i/r_i \rightarrow \infty$ and, again, $R_i^{\alpha} > r_i$ for each $i$. $\beta$ will be fixed later, for now it is just a real greater than 1. As the process is $\alpha$-scaling one can map $Rec(x_i,R_i^{\alpha},R_i)$ onto the unit square via an affine map $T$ and the image of the graph under this transformation, denoted $G_{X_i}^{[0,1]}$, will have distribution $X_i$ equal to the original distribution itself as it is scaled following the definition of $\alpha$-scaling. Therefore the probability of an event $E_i$ is equal to the probability the graph of $X_i$ stays in the unit square and 
$$Rec(1/2,r_i/R_i^{\alpha} ,r_i^{1/\alpha}/R_i ) \cap G_{X_i}^{I(1/2, r_i/R_i^{\alpha})} = G_{X_i}^{I(1/2, r_i^{\beta}/R_i^{\alpha})}.$$

Thus the probability of $E_i$ depends solely on the ratio $R_i/r_i = q_i$. If $\sum P(E_i)$ diverges then the conditions for Borel-Cantelli are satisfied and the argument continues. However, if not, the sequence $r_i$ is modified in the following way. Each $i$ gives us a ratio $q_i$ and a probability $P(E_i)$. Construct a function $g \colon \mathbb{N} \rightarrow \mathbb{N}$ such that $g(n) = \lceil \frac{1}{nP(E_n)}\rceil$ for all $n\in \mathbb{N}$. Then, keeping $R_i$ fixed, change the $r_i$ so that each ratio $q_i$ is repeated $g(i)$ many times. For instance, if $g(1) = 3$ then $r_1,r_2$ and $r_3$ are chosen so that $R_1/r_1, R_2/r_2$ and $R_3/r_3$ all give the same $P(E_1)$ and $r_4$ then is chosen with respect to $P(E_2)$ etc. The new sequence is constructed such that $\sum P(E_i)$ diverges, satisfying the conditions for Borel-Cantelli.

Hence, by the Borel-Cantelli lemma, inifinitely many $E_i$ occur with probability one. So there are sequences $x_i \in [0,1]$, $R_i > r_i > 0$ and $\beta > 1$ such that, with full probability, all of the events $E_i$ happen. 

Given a specific event $E_i$ we wish to consider the measure of the rectangles. The ratio of measures of such rectangles is determined by the original measure on the interval. We denote the lower regularity dimension of $\mu$ by $t$, recall this is strictly positive due to Proposition 2.1 and obtain the following bound:
\[
\frac{f_*\mu(Rec(x_i,R_i^{\alpha},R_i))}{f_*\mu(Rec(x_i,r_i,r_i^{1/\alpha}))} = \frac{\mu(B(x_i, R_i^{\alpha}))}{\mu(B(x_i, r_i^{\beta}))} \ge C\left(\frac{R_i^{\alpha}}{r_i^{\beta}}\right)^t, 
\]
where $C$ comes from the definition of the lower regularity dimension.

The only variable left to be fixed is $\beta$. We wish to have the above ratio greater than $C(R_i/r_i)^s$. After a short calculation, it is clear that this is always true if $\beta \ge \alpha$. Thus by choosing such a $\beta$ we have
\[
\frac{f_*\mu(Rec(x_i,R_i^{\alpha},R_i))}{f_*\mu(Rec(x_i,r_i,r_i^{1/\alpha}))} \ge C\left(\frac{R_i}{r_i}\right)^s. 
\]

To show the upper regularity dimension is greater than $s$ we need to consider balls not rectangles. Thankfully due to our construction $B(x_i,R_i) \supset Rec(x_i, R_i^\alpha, R_i)$ and $B(x_i,r_i) \subseteq Rec(x_i, R_i, R_i^{1/\alpha})$. Hence
\[
\frac{f_*\mu(B(x_i,R_i))}{f_*\mu(B(x_i,r_i))} \ge \frac{f_*\mu(Rec(x_i,R_i^{\alpha},R_i))}{f_*\mu(Rec(x_i,r_i,r_i^{1/\alpha}))} \ge C\left(\frac{R_i}{r_i}\right)^s ,
\]
completing the proof.


For the lower regularity dimension it suffices to change the events $E_i$. Assuming $\alpha>1$, let $x_i \in [0,1]$, $R_i > r_i > 0$ and $\beta < 1$, then $E_i$ is the event where $G_X^{I(x_i, R_i)} \cap Rec(x_i,R_i,R_i^{1/\alpha}) \subseteq Rec(x_i, r_i^{\beta}, R_i^{1/\alpha})$ and $G_X^{I(x_i, r_i)} \subseteq Rec(x_i, r_i^{\alpha}, r_i)$. The previous argument then works in much the same way, showing that the lower regularity dimension of $f_*\mu$ is zero as desired.






\subsection{$\alpha$-decaying and uniformly perfect measures}





\begin{proof}[Proof of Proposition 2.8]
If $\mu$ is $\alpha$-decaying then $\mu(B(x,\varepsilon R)) \le C \varepsilon^{\alpha} \mu(B(x,R))$ for any $x\in X$ and $\varepsilon, R >0$. Thus
\[
\frac{\mu(B(x,R))}{\mu(B(x,\varepsilon R))} \ge \frac{1}{C} \left(\frac{R}{\varepsilon R} \right)^{\alpha}
\]
and so $\lrdim \mu \ge \alpha$.

For the other direction, assume $\lrdim \mu = t$. Then for any $\delta > 0$, $x\in X$ and $R>r>0$ 
\begin{align*}
    \frac{\mu(B(x,R))}{\mu(B(x,r))} &\ge C \left( \frac{R}{r}\right)^{t - \delta}.
\end{align*}
Given $\varepsilon > 0$ and $R > 0$, choose $r = \varepsilon R$. Inserting this value of $r$ into the above yields 
\[\mu(B(x,R))) \ge C \left(\frac{R}{\varepsilon R}\right)^{t-\delta} \mu(B(x,\varepsilon R)).
\]
Hence
\[
\mu(B(x,\varepsilon R)) \le \frac{1}{C} \varepsilon^{t - \delta} \mu(B(x,R))  
\]
as desired, with $\alpha = t$.
\end{proof}






\section*{Acknowledgements}



\bibliography{bib-main}

\bibliographystyle{plainnat}

%%%%%%%%%%%%%%%%%%%%%%%%%%%%%%%%%%%%%%%%
\begin{comment}
\begin{thebibliography}{99}




\bibitem[B]{bedford}
T. Bedford.
{\em Crinkly curves, Markov partitions and box dimension in self-similar sets},
Ph.D. dissertation, University of Warwick, (1984).

\bibitem[DS]{das-simmons}
T. Das and D. Simmons. 
The Hausdorff and dynamical dimensions of self-affine sponges: a dimension gap result,
\emph{Invent. Math. (to appear)}, (2016), available at: https://arxiv.org/abs/1604.08166.

\bibitem[F]{falconer}
K.~J. Falconer.
{\em Fractal Geometry: Mathematical Foundations and Applications},
2nd Ed., John Wiley, Hoboken, NJ, (2003).

\bibitem[Fr]{fraser}
J. M. Fraser.
Assouad type dimensions and homogeneity of fractals,
\emph{Trans. Amer. Math. Soc.}, {\bf366}, (2014), 6687--6733.

\bibitem[FH]{fraser-howroyd}
J. M. Fraser and D. C. Howroyd. 
Assouad type dimensions of self-affine sponges,
\emph{Ann. Acad. Sci. Fenn. Math.}, \textbf{42}, (2017), 149--174.

\bibitem[FJ]{fraser-jordan}
J. M. Fraser and T. Jordan. 
The Assouad dimension of self-affine carpets with no grid structure,
\emph{Proc. Amer. Math. Soc.}, \textbf{145}, (2017), 4905--4918.

\bibitem[Fu]{furstenberg}
H. Furstenberg.
 Ergodic fractal measures and dimension conservation,
\emph{Ergodic Theory Dynamical Systems}, {\bf 28}, (2008), 405--422.

\bibitem[H]{hochman}
M. Hochman.
Dynamics on fractals and fractal distributions,
\emph{preprint}, 2010, available at http://arxiv.org/abs/1008.3731.

\bibitem[JJKRRS]{kaenmakinew}
E. J\"arvenp\"a\"a, M. J\"arvenp\"a\"a, A. K\"aenm\"aki, T. Rajala, S. Rogovin and V. Suomala.
Packing dimension and Ahlfors regularity of porous sets in metric spaces,
\emph{Math. Z.}, { \bf 266}, (2010),   83--105.


\bibitem[KL]{anti2}
A. K{\"a}enm{\"a}ki and J. Lehrb{\"a}ck.
Measures with predetermined regularity and inhomogeneous self-similar sets,
\emph{preprint}, (2016), available at: https://arxiv.org/abs/1609.03325.

\bibitem[KLV]{anti1}
A. K{\"a}enm{\"a}ki, J. Lehrb{\"a}ck and M. Vuorinen.
Dimensions, Whitney covers, and tubular neighborhoods,
{\em Indiana Univ. Math. J.}, {\bf 62} No.6, (2013), 1861–1889.

\bibitem[KP]{kenyonperes}
R. Kenyon and Y. Peres.
Measures of full dimension on affine-invariant sets,
\emph{Ergodic Theory Dynam. Systems}, {\bf 16}, (1996), 307--323.

\bibitem[KV]{konyagin}
S.V. Konyagin and A.L. Vol'berg.
On measures with the doubling condition, \emph{Math. USSR-Izv.}, \textbf{30}, (1988), 629--638.

\bibitem[LWW]{doublingcarpets}
H. Li, C. Wei and S. Wen.
Doubling Property of Self-affine Measures on Carpets of Bedford and McMullen,
 \emph{Indiana Univ. Math. J. ({\em to appear})}, available at: http://www.iumj.indiana.edu/IUMJ/Preprints/5826.pdf

\bibitem[LS]{luksak}
J. Luukkainen and E. Saksman.
Every complete doubling metric space carries a doubling measure, \emph{Proc. Amer. Math. Soc.}, \textbf{126}, (1998), 531--534.

\bibitem[M]{mackay}
J. M. Mackay.
Assouad dimension of self-affine carpets,
\emph{Conform. Geom. Dyn.}, {\bf 15}, (2011), 177--187.

\bibitem[MT]{mackaytyson}
J. M. Mackay and J. T. Tyson.
\emph{Conformal dimension. Theory and application},
University Lecture Series, 54. American Mathematical Society, Providence, RI, (2010).

\bibitem[Ma]{mattila}
P. Mattila.
{\em Geometry of Sets and Measures in Euclidean Spaces},
Cambridge University Press, London, (1999).

\bibitem[Mc]{mcmullen}
C.T. McMullen.
The Hausdorff dimension of general Sierpi\'nski carpets,
{\em Nagoya Math. J.}, {\bf 96}, (1984), 1--9.


\bibitem[O]{ojala}
T. Ojala.
{\em Thin and Fat sets: geometry of doubling measures in metric spaces},
Ph.D. dissertation, University of Jyväskylä, (2014).

\bibitem[Ol1]{olsenformalism}
L. Olsen.
A Multifractal Formalism,
\emph{Advances in Mathematics}, {\bf 116} Issue 1, (1995), 82--196.

\bibitem[Ol2]{sponges}
L. Olsen.
Self-affine multifractal Sierpinski sponges in $\mathbb{R}^d$,
{\em Pacific J. Math}, {\bf 183}, (1998), 143--199.

\bibitem[P]{preiss}
D. Preiss.
Geometry of measures in $\mathbb{R}^n$: Distribution, rectifiability, and densities,
{\em Ann. of Math.}, {\bf  125},  (1987), 537--643.

\bibitem[R]{robinson}
J. C. Robinson.
{\em Dimensions, Embeddings, and Attractors},
Cambridge University Press, Cambridge, (2011).




\end{thebibliography}
\end{comment}



\end{document}            