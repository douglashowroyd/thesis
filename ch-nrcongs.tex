\chapter{The number of congruences of a semigroup}
\label{chap:nrcongs}

Introduction

How many congruences does a semigroup have? %TODO: proper introduction

\section{Congruence-full semigroups}
% TODO: this section needs a full rethink of infinite semigroups

We start by giving the definition of a \textit{congruence-full} semigroup, by
analogy with a \textit{congruence-free} semigroup \cite[3.7]{howie}.

\begin{definition}
  A semigroup is \textbf{congruence-full} if every equivalence relation defined
  on it is a congruence.
\end{definition}

Since the number of equivalences on a set is given by the sequence of Bell
numbers $(B_n)_{n \in \mathbb{N}}$, we can say that a semigroup is
congruence-full if and only if it has precisely $B_n$ congruences, where $n$ is
the size of the semigroup.
% TODO: semigroup might be infinite!  No such thing as n.

Finite congruence-free semigroups are classified in \cite[3.7]{howie}.
In this section we explore finite congruence-full semigroups, culminating in a
complete classification in Theorem \ref{thm:congruence-full}.  First we need to
build up some knowledge about the Green's relations of congruence-full
semigroups.

\begin{lemma}
  \label{lem:m-is-h-trivial}
  A congruence-full semigroup of size greater than $2$ has $\HH$-trivial
  minimal ideal.
  % TODO: infinite?  Does it have to have a minimal ideal?
  \begin{proof}
    Let $S$ be a congruence-full semigroup with more than $2$ elements, and let
    $M$ be its minimal ideal.  Since $M$ is simple, every $\HH$-class of $M$ is
    a group.  We will proceed by considering possible sizes of $M$'s
    $\HH$-classes, and showing that any size greater than $1$ violates the
    assumption that $S$ is congruence-full.

    Firstly, let $H$ be an $\HH$-class in $M$ with at least $3$ elements.  Let
    $1_H$ be the group identity of $H$, and let $g,h \in H\setminus\{1_H\}$
    with $g \neq h$.  Now let $\sim$ be $(1_H, g)^e$.  Since $g$ is not the
    identity, we know that $gh \neq h$.  Hence we have $1_H \sim g$ but
    $1_H h \nsim gh$, so $\sim$ is not a congruence.  Hence $S$ is not
    congruence-full, a contradiction.

    Instead, let $H$ be an $\HH$-class in $M$ with precisely $2$ elements.
    Since $|S| \geq 3$ we know that $S \setminus H$ is non-empty.  If there
    exists some $x \in S \setminus M$, then let $h \in H \setminus \{1_H x\}$,
    and let $\sim$ be $(x,h)^e$; since $h \neq 1_H x$ we have $x \sim h$ but
    $1_H x \nsim 1_H h$, so $\sim$ is not a congruence.  If on the other hand
    $S \setminus M$ is empty, then $M$ must contain an $\HH$-class other than
    $H$.  Choose some $x \in M \setminus H$ such that $x \LL 1_H$ (if this is
    not possible, we can choose $x$ such that $x \RR 1_H$, and a similar
    argument holds).  Let $h \in H \setminus \{1_H\}$, and let $\sim$ be
    $(x,1_H)^e$.  We have $xh \RR x \nRR h$, so $xh \nRR h$ and in particular
    $xh \neq h$.  Hence $x \sim 1_H$ but $xh \nsim 1_H h$, so $\sim$ is
    not a congruence.  Either of these cases violates the assumption that $S$
    is congruence-full, a contradiction.
  \end{proof}
\end{lemma}

\begin{lemma}
  \label{lem:m-is-l-or-r-trivial}
  A congruence-full semigroup of size greater than $2$ has a minimal ideal which
  is either $\LL$-trivial or $\RR$-trivial.
  \begin{proof}
    Let $S$ be a congruence-full semigroup with more than $2$ elements, with a
    minimal ideal $M$ which is neither $\LL$-trivial nor $\RR$-trivial.

    We know by Lemma \ref{lem:m-is-h-trivial} that $M$ is $\HH$-trivial.
    Since $M$ is simple and $\HH$-trivial, it is a rectangular band.  Let
    $x_{11}, x_{12}, x_{22} \in M$ be pairwise distinct elements with
    $x_{11} \RR x_{12} \LL x_{22}$, and let $\sim$ be the relation
    $(x_{11},x_{22})^e$.  Since $M$ is a rectangular band, we have
    $x_{11}x_{22} = x_{12}$ and $x_{11}x_{11} = x_{11}$.  Hence
    $x_{11} \sim x_{22}$ but $x_{11}x_{11} \nsim x_{11}x_{22}$, and so $\sim$ is
    not a congruence.  This means that $S$ is not congruence-full, a
    contradiction.
  \end{proof}
\end{lemma}

\begin{lemma}
  \label{lem:simple-or-zero-semigroup}
  A finite congruence-full semigroup of size greater than $2$ is either simple
  or a zero semigroup.
  \begin{proof}
    Let $S$ be a congruence-full semigroup with more than $2$ elements, with
    minimal ideal $M$.  Let us assume $S$ is not simple; this means that
    $S \setminus M$ is non-empty.  By Lemma \ref{lem:m-is-l-or-r-trivial}, $M$
    is either $\LL$-trivial or $\RR$-trivial; without loss of generality let us
    assume that $M$ is $\LL$-trivial (a similar argument applies for
    $\RR$-triviality).  We will start by proving that $S$ contains a zero, and
    then we will go on to prove that $S$ is a zero semigroup.
    Firstly, aiming for a contradiction, let us assume that $|M| > 1$.

    If $S \setminus M$ contains an idempotent, call it $x$.  Choose $m,n \in M$
    with $m \neq n$.  Now, either $mx = m$ or $mx \neq m$.  If $mx = m$, then
    let $\sim$ be $(n,x)^e$: since by $\LL$-triviality $mn = n$, and since
    $mx = m$, we have $n \sim x$ but $mn \nsim mx$, so $\sim$ is not a
    congruence, a contradiction.  If on the other hand $mx \neq m$, then let
    $\sim$ be $(m,x)^e$: since $x \neq mx \neq m$ and $xx=x$, we have $m \sim x$
    but $mx \nsim xx$, so $\sim$ is not a congruence, a contradiction.

    If instead, $S \setminus M$ does not contain an idempotent, then there must
    exist some $x \in S \setminus M$ such that $x^2 \in M$.
    \footnote{Does this fact need proof?  Does it hold for infinite semigroups?
    If so, I think we can remove the word ``finite'' everywhere and the final
    theorem is much stronger.}
    Let $m \in M \setminus \{x^2\}$ (this is possible since $|M| > 1$) and let
    $\sim$ be $(m,x)^e$.  Since by $\LL$-triviality $xm = m$, we have $m \sim x$
    but $xm \nsim xx$, so $\sim$ is not a congruence, a contradiction.

    We have now shown that $|M| > 1$ violates the condition that $S$ is
    congruence-full.  Hence the minimal ideal $M$ must contain precisely one
    element, $0$: we have $0x = x0 = 0$ for any $x \in S$, so $0$ is a zero for
    $S$. Next we will show that $S$ is a zero semigroup, i.e.~that $xy = 0$ for
    all $x,y \in S$.  Clearly if $x$ or $y$ is $0$ then $xy = 0$.

    Let $x,y \in S \setminus \{0\}$ with $x \neq y$.  The product $xy$ cannot be
    equal to both $x$ and $y$, so without loss of generality let us assume that
    $xy \neq x$.  Assume, aiming for a contradiction, that $xy \neq 0$.  Let
    $\sim$ be the relation $(x,0)^e$; since $0y = 0$ and $xy \neq x$ we have
    $x \sim 0$ but $xy \nsim 0y$, so $\sim$ is not a congruence, a
    contradiction.  Hence for distinct $x,y \in S$ we have $xy = 0$.

    It only remains to consider whether $x^2=0$ for every $x \in S$.  Let
    $x \in S \setminus \{0\}$ and assume, aiming for a contradiction, that
    $x^2 \neq 0$.  Let $y \in S \setminus \{0,x\}$ (possible since $|S| > 2$)
    and let $\sim$ be $(x,y)^e$; since $xy=0$ but $x^2 \neq 0$, we have
    $x \sim y$ but $xx \nsim xy$, so $\sim$ is not a congruence, a
    contradiction.  Hence $xy = 0$ for all $x,y \in S$, so $S$ is a zero
    semigroup.
  \end{proof}
\end{lemma}

We can now state the main theorem of this section, a classification of all the
finite congruence-full semigroups.

\begin{theorem}
  \label{thm:congruence-full}
  A finite semigroup is congruence-full if and only if it is a zero semigroup, a
  left zero semigroup, a right zero semigroup, or has size less than or equal to
  $2$.
  \begin{proof}
    Let $S$ be a finite congruence-full semigroup of size greater than $2$.  If
    $S$ is not simple, then by Lemma \ref{lem:simple-or-zero-semigroup} it is a
    zero semigroup.  If $S$ is simple, it is equal to its minimal ideal.
    Therefore, by Lemma \ref{lem:m-is-l-or-r-trivial}, $S$ is either
    $\LL$-trivial or $\RR$-trivial.  If it is $\LL$-trivial then $xy=y$ for all
    $x,y \in S$, so it is a right zero semigroup.  If it is $\RR$-trivial then
    $xy=x$ for all $x,y \in S$, so it is a left zero semigroup.

    To prove the converse, we consider zero, left zero, and right zero
    semigroups in turn.  First, let $S$ be a zero semigroup and let $\sim$ be an
    equivalence relation on $S$.  Let $x,y,s,t \in S$ such that $x \sim y$ and
    $s \sim t$.  We have $xs = 0 = yt$, so $xs \sim yt$ and therefore $\sim$ is
    a congruence.  Hence $S$ is congruence-full.

    Alternatively, let $S$ be a left zero semigroup and let $\sim$ be an
    equivalence relation on $S$.  Let $x,y,a \in S$ such that $x \sim y$.  We
    have $ax = a = ay$ and $xa = x \sim y = ya$, so $ax \sim ay$ and
    $xa \sim ya$.  Hence $\sim$ is a congruence, so $S$ is congruence-full.
    A similar argument proves the statement for right zero semigroups.

    % Mention size 2
  \end{proof}
\end{theorem}



\section{Semigroups with fewer congruences}

If a semigroup of size $n$ is not congruence-full, it has fewer than $B_n$
congruences.  For some low values of $n$ a semigroup may have $B_n - 1$
congruences, but as $n$ grows the number of congruences seems to be lower.  In
this section we propose a value for the ``second highest'' number of congruences
on a semigroup of size $n$.

\begin{conjecture}
  \label{conj:not-cong-full}
  A semigroup of size $n > 3$ which is not congruence-full has at most
  $2B_{n-1}$ congruences.
\end{conjecture}

This conjecture, which does not yet have a proof, is supported by experimental
investigation.  An exhaustive analysis of all semigroups up to isomorphism and
anti-isomorphism shows that the conjecture holds for $n \leq 7$, and also
reveals a pattern in the semigroups which attain the limit.  This pattern is
stated in the next conjecture.

\begin{conjecture}
  \label{conj:cong-nearfull-7}
  Let $n > 3$.  There are precisely 7 semigroups (up to isomorphism and
  anti-isomorphism) of size $n$ which have $2B_{n-1}$ congruences.
\end{conjecture}

The 7 semigroups are described in the following table, and a diagram is given
for $n = 4$.  In the descriptions, when $\Z_{n-1}$ is a subsemigroup of $S$, the
zero of $\Z_{n-1}$ is called $z$.

\begin{longtable}{| c | >{\centering}m{0.5\linewidth} | m{0.3\linewidth} |}
  \hline
  No. & Diagram of $S$ & Description of $S$
  \\ \hline
  1 & \includesvg{pics/ch-nrcongs/lz-plus-id}
  & Left-zero semigroup $\LZ_{n-1}$ with an idempotent $c$ appended.
  There is a distinguished $p \in M$ such that $cx=p$ and $xc=x$ for
  all $x \in M$.
  \\ \hline
  2 & \includesvg{pics/ch-nrcongs/lz-plus-nonid}
  & Left-zero semigroup $\LZ_{n-1}$ with a non-idempotent element $c$
  appended.
  Multiplication defined by $cx=c^2$ and $xc=x$ for all $x \in M$.
  \\ \hline
  3 & \includesvg{pics/ch-nrcongs/z-plus-zero}
  & Zero semigroup $\Z_{n-1}$ with a zero appended.
  \\ \hline
  4 & \includesvg[scale=0.8]{pics/ch-nrcongs/z-plus-id} % TODO: consistent scale
  & Zero semigroup $\Z_{n-1}$ with an idempotent $c$ appended above the minimal
  ideal.
  Multiplication defined by $cx=xc=z$ for all $x \in \Z_{n-1}$.
  \\ \hline
  5 & \includesvg{pics/ch-nrcongs/z-plus-m-id}
  & Zero semigroup $\Z_{n-1}$ with an idempotent $c$ appended in the minimal
  ideal.
  Multiplication defined by $cx=c$ and $xc=z$ for all $x \in \Z_{n-1}$.
  \\ \hline
  6 & \includesvg{pics/ch-nrcongs/c2-plus-nonids}
  & Zero semigroup $\Z_{n-1}$ with an element $c$ appended in the same
  $\HH$-class as $z$.
  Multiplication defined by $c^2=z$ and $xc=cx=c$ for all $x \in \Z_{n-1}$.
  \\ \hline
  7 & \includesvg{pics/ch-nrcongs/c2-plus-nonids}
  & Cyclic group $C_2 = \{\id, g\}$ with $n-2$ elements appended such that
  $xy=\id$, $x \id=\id x=g$ (for all $x,y \in S \setminus C_2$).
  \\ \hline
\end{longtable}

Of the 7 semigroups in the table above, semigroups 1 to 6 contain a copy of
either $\Z_{n-1}$ or $\LZ_{n-1}$ as a subsemigroup.  The one element outside
this subsemigroup, which we will call $c$ (the \textit{child element}) is key to
understanding why these semigroups have $2B_{n-1}$ congruences.  In each
semigroup, there is another element $p$ (the \textit{parent element}) such that
an equivalence $\sim$ is a congruence if and only if $c \sim p$ or $[c]_\sim$ is
a singleton.  In other words, the child has to be alone or with its parent.
Now, since $S \setminus \{c\}$ is congruence-full, we can take any congruence
(any equivalence) $\sim$ on $S \setminus \{c\}$ and extend it to a congruence on
$S$ in two different ways: by including $c$ as a singleton, or by including $c$
in the same congruence class as $p$.  Since there are $B_{n-1}$ choices for
$\sim$, this gives us precisely $2B_{n-1}$ congruences on $S$.

Semigroup 7 is unique in that it does not contain $\Z_{n-1}$ or $\LZ_{n-1}$ as a
subsemigroup.  However, it still fulfils the \textit{child/parent} condition
above, where $c = \id$ and $p = g$.

\section{Other findings}
We applied the methods from Chapter \ref{chap:lattice} to a bunch of semigroups,
including the partition monoids and small semigroups described elsewhere in this
thesis.

Looks like most congruences are principal?

Show some pictures of lattices on famous semigroups.
