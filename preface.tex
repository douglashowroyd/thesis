\chapter*{Preface}
\addcontentsline{toc}{chapter}{Preface}

Semigroup theory has its roots in group theory.  Groups were used in some form
as early as the late 1700s by Lagrange, in order to solve equations; this work
was continued in the early 1800s by Galois, who first used the word
\textit{group} to describe them.  Groups were then studied in various different
contexts -- in geometry, in number theory, and as permutation groups -- for some
time before the various branches of theory were united into one, with von Dyck
inventing the modern abstract definition of a group in 1882 \cite{dyck_1882}.
The study of group theory has flourished since then, becoming a major area of
research in pure mathematics.  By contrast, semigroup theory is a relatively
young area of study, having been defined only in the early 1900s and having been
studied very little before the 1950s, other than a few early papers by authors
such as Suschkewitch \cite{susch_1928}.  However, in the second half of the
twentieth century, semigroup theory gained traction, and now accounts for a
significant body of work, with dedicated journals such as Semigroup Forum, and
seminal books such as \cite{howie} and \cite{petrich}.

Since any group is a semigroup, it might be imagined that groups would be
studied simply as a special case within semigroup theory.  In practice, however,
group theory tends to inform semigroup theory, since it turns out that groups
are a very important topic within the study of generic semigroups.  Important
features of a semigroup's structure are determined by its maximal subgroups --
for example their Green's relations, or the linked triples on a completely
simple semigroup.  As a result, groups are a central part of semigroup theory,
allowing us to borrow from the richly developed field of group theory in order
to solve problems for semigroups in general.

Computational algebra has existed for nearly as long as the theory of
computation itself.  For instance, perhaps the earliest published group theory
algorithm is that by Dehn \cite{dehn_1911} for solving the word problem in
certain groups.  The Todd--Coxeter algorithm \cite{todd_coxeter_1936}, which
enumerates the cosets of a subgroup in a group, was certainly designed to be
carried out by hand, having been described in the 1930s before the invention of
electronic computers.  When computers did arrive, there was early interest in
using them for group theory problems, with Todd--Coxeter being implemented on
the EDSAC II in Cambridge as early as 1953 \cite{leech_1963}.  Since then,
computational group theory has flourished, with a variety of software packages
such as \textsf{Magma} \cite{magma}, \ACE{} \cite{ace}, and particularly \GAP{}
\cite{gap}, which has a variety of packages containing algorithms to solve a
wide range of problems.  A wealth of material is available on computational
group theory, including several dedicated books \cite{sims, cgt}.  By
comparison, computational semigroup theory is much younger and less developed,
as is semigroup theory as a whole.  Computers were used as early as 1953 to
classify semigroups of low order \cite{tamura_1953, froidure_pin}, but the
theory of computing with semigroups developed more slowly than with groups, and
no package emerged for semigroups on the scale of the group algorithms in
\GAP{}.  However, there has been considerable interest in computational
semigroup theory in recent years, and increasingly there do exist algorithms for
semigroups, as well as software packages implementing them, such as
\Semigroupe{} \cite{semigroupe}, \libsemigroups{} \cite{libsemigroups}, and the
\GAP{} packages \Semigroups{} \cite{semigroups}, \smallsemi{} \cite{smallsemi}
and \kbmag{} \cite{kbmag}.

In the same way that semigroup theory borrows results from group theory,
computational semigroup theory often borrows algorithms from computational group
theory.  The Todd--Coxeter and Knuth--Bendix algorithms, for instance, both of
which were originally applied to groups, extend naturally to semigroups, and to
semigroup congruences, as will be discussed in this thesis.  Furthermore, when
we use the maximal subgroups of a semigroup, we can use group algorithms to
compute information about the semigroup.  For example, when computing the linked
triples on a completely simple semigroup, we can use group theory to find all
the normal subgroups of a given maximal subgroup.  In this way, computational
group theory is not just a subset, but a key part, of computational semigroup
theory.

This thesis deals primarily with the congruences on a semigroup.  A semigroup's
congruences describe its homomorphic kernels and images -- that is, the ways in
which the semigroup can be mapped onto another semigroup while preserving the
semigroup operation.  In this way, a semigroup's congruences serve the same
function as a group's normal subgroups, or a ring's two-sided ideals, though the
definition is more abstract and general than either of these.  The congruences
of several famous semigroups have been classified previously -- for example, the
full transformation monoid $\T_n$ by Mal$'$cev \cite{malcev_1952}, the symmetric
inverse monoid $\I_n$ by Liber \cite{liber_1953} and the partial transformation
monoid $\PT_n$ by Shutov \cite{shutov_1988} -- and in recent years a host of
other monoids of partial transformations, restricted variously to elements that
preserve or reverse the order or orientation of the set being acted on, have
also had their congruences classified \cite{fernandes_2000, lisbon_ii,
  lisbon_i}.  Classifying the congruences of other important semigroups is a
major activity in the study of congruences, and will form the majority of Part
\ref{part:results} of this thesis.

The computational theory of semigroup congruences is also a young field.  Some
algorithms for computing information about a given congruence have existed in
the \GAP{} library for many years, but many of them lack sophistication, being
close to brute-force.  In some cases, we can do little to improve on these
algorithms, since semigroup congruences in general do not have the regular
structure of their group theory or ring theory counterparts.  However, in other
cases we can improve on these using existing knowledge about specific categories
of semigroup.  Firstly we can do this by considering alternative structures that
correspond to a semigroup's congruences.  For instance, in group theory we
consider a group's normal subgroups rather than studying its congruences
directly; in the same way, we can study a completely simple semigroup's linked
triples, or an inverse semigroup's kernel--trace pairs, in place of their
congruences, and thus we can often produce the answers to computational
questions more quickly than by using direct methods.  This approach forms the
basis of the present author's previous works \cite{mtorpey_pre_msc,
  mtorpey_msc}, which are expanded upon in this thesis. In the more general
cases, where no such alternative structure exists, we may still make
improvements on existing algorithms by applying new algorithms that are not
currently used for congruences.  We will take both these approaches in Part
\ref{part:algorithms} of this thesis, presenting new congruence algorithms, as
well as showing ways in which existing algorithms can be successfully adapted
for congruence-related purposes.

After introducing some preliminary theory, this thesis is divided into two broad
parts.  Part \ref{part:algorithms} discusses the computational theory of
congruences, and presents algorithms that can be used to answer
congruence-related questions.  Part \ref{part:results} shows some results that
have been obtained using these algorithms, as well as results that have been
proven by hand using computational output as a starting point.

Chapter \ref{chap:intro} acts as an introduction to this document, providing the
preliminary knowledge which is required to understand the material in the rest
of the thesis.  It is mostly concerned with ideas from semigroup theory such as
Green's relations, generators, congruences and presentations.  It also
introduces several important types of elements that form semigroups, such as
partial transformations and bipartitions.  Later, some computational issues such
as algorithms and decidability are covered, as well as some algorithms that are
used later on.

Chapter \ref{chap:pairs} presents a new way of computing with congruences
defined by generating pairs.  The method presented uses parallel computation to
run a variety of algorithms -- including Todd--Coxeter, Knuth--Bendix, and an
unsophisticated algorithm known as \textit{pair orbit enumeration} -- that test
whether a given pair lies in a congruence, given its generating pairs.  This
approach takes advantage of the different algorithms' abilities to return an
answer quickly in various cases, in each case exhibiting run-times close to the
minimum of all the different algorithms.  Modifications for left and right
congruences are explained, and different versions are described for finitely
presented semigroups and for \textit{concrete} semigroups (finite semigroups of
transformations or other similar elements).  This approach was implemented in
\libsemigroups{} \cite{libsemigroups}, and the results of benchmarking tests on
that implementation are shown near the end of the chapter.

Chapter \ref{chap:converting} concerns the various ways of representing a
congruence, other than as a set of pairs.  Five possible representations are
described -- generating pairs, normal subgroups, linked triples, kernel--trace
pairs, and ideals -- along with explanations of the precise semigroups and
congruences to which they apply.  Algorithms are given for converting from one
representation to another, so that a computational algebra system may quickly
convert a congruence specified in a certain way to the most efficient
representation, and thus answer questions about the congruence in as short a
time as possible.  All twenty of the possible conversions between the five
representations are considered: many of these are currently implemented in
\Semigroups{} \cite{semigroups}, some having never been described before; others
are trivial or unnecessary; and two remain open problems.

Chapter \ref{chap:lattice} presents an algorithm for computing the entire
congruence lattice of a finite semigroup.  This algorithm is rather rudimentary,
but various shortcuts and improvements are described to try to reduce the
computational time required, where possible.  A version of this algorithm is
implemented in \Semigroups{}, and thus takes advantage of the methods described
in Chapters \ref{chap:pairs} and \ref{chap:converting}.  This makes it possible
to compute the lattices of many sufficiently small semigroups in a reasonable
time; a brief analysis is included of the sizes of semigroup and lattice which
are feasible, along with some visual examples of lattices that have been
computed using this method.

Chapter \ref{chap:motzkin} considers the Motzkin monoid $\Mot_n$, the monoid of
all planar bipartitions of degree $n$ with blocks of size no greater than $2$.
An investigation into the congruence lattice of this monoid was initiated by
computational experiments, using the method described in Chapter
\ref{chap:lattice}, and resulted in a complete classification of the congruence
lattice of $\Mot_n$ for arbitrary $n$, along with generating pairs for each
congruence.  This classification, originally published in \cite{ourpaper}, is
shown here with the kind permission of my co-authors.  Other important monoids
of bipartitions are also considered, and their congruence lattices classified.

Chapter \ref{chap:other} completes this thesis by presenting some other results
obtained or precipitated by computational experiments in the \Semigroups{}
package.  Firstly, we consider the congruences on the principal factors of the
full transformation monoid $\T_n$ and some other related monoids, classifying
them for arbitrary $n$.  Secondly, we consider the number of congruences that
exist on an arbitrary finite semigroup.  We give some upper and lower bounds for
this number based on a semigroup's size, and we present two conjectures about
the second-largest number of congruences a finite semigroup can have.  To
support these conjectures, we also show some computational evidence produced
with the aid of \smallsemi{} \cite{smallsemi}.  Finally, we present some
findings from an exhaustive classification of the congruences on all $853303$
semigroups of size no larger than $7$, up to isomorphism and anti-isomorphism.

At the end of this document we provide an index of the various terms that are
used.  We also provide a list of notation, with a brief description of what each
mathematical symbol means.  In both of these, each entry has a reference to the
page on which the term or symbol is first defined.  In the original digital
version of this document, all citations and numbered references act as
hyperlinks, allowing the reader to click them to be redirected to the
appropriate location.
