\chapter*{Preface}
\addcontentsline{toc}{chapter}{Preface}

Semigroup theory has its roots in group theory.  Groups were used in some form
as early as the late 1700s by Lagrange, in order to solve equations; this work
was continued in the early 1800s by Galois, who first used the word
\textit{group} to describe them.  Groups were then studied in various different
contexts -- in geometry, in number theory, and as permutation groups -- for some
time before the various branches of theory were united into one, with von Dyck
inventing the modern abstract definition of a group in 1882 \cite{dyck_1882}.
The study of group theory has flourished since then, becoming a major area of
research in pure mathematics.  By contrast, semigroup theory is a relatively
young area of study, having been defined only in the early 1900s and having been
studied very little before the 1950s, other than a few early papers by authors
such as Suschkewitch \cite{susch_1928}.  However, in the second half of the
twentieth century, semigroup theory gained traction, and now accounts for a
significant body of work, with dedicated journals such as Semigroup Forum, and
seminal books such as \cite{howie} and \cite{petrich}.

Since any group is a semigroup, it might be imagined that groups would be
studied simply as a special case within semigroup theory.  In practice, however,
group theory tends to inform semigroup theory, since it turns out that groups
are a very important topic within the study of semigroups in general.  Important
features of a semigroup's structure depend on its maximal subgroups --
for example its Green's relations, or in the case of a completely simple
semigroup, its linked triples.  As a result, groups are a central part of
semigroup theory, allowing us to borrow from the richly developed field of group
theory in order to solve problems for semigroups that are not groups.

Computational algebra has existed for nearly as long as the theory of
computation itself.  For instance, perhaps the earliest published group theory
algorithm is that by Dehn \cite{dehn_1911} for solving the word problem in
certain groups.  The Todd--Coxeter algorithm \cite{todd_coxeter_1936}, which
enumerates the cosets of a subgroup in a group, was certainly designed to be
carried out by hand, having been described in the 1930s before the invention of
electronic computers.  When computers did arrive, there was early interest in
using them for group theory problems, with Todd--Coxeter being implemented on
the EDSAC II in Cambridge as early as 1953 \cite{leech_1963}.  Since then,
computational group theory has flourished, with a variety of software packages
such as \textsf{Magma} \cite{magma}, \ACE{} \cite{ace}, and particularly \GAP{}
\cite{gap}, which has a variety of packages containing algorithms to solve a
wide range of problems.  A wealth of material is available on computational
group theory, including several dedicated books \cite{sims, cgt}.  By
comparison, computational semigroup theory is much younger and less developed,
as is semigroup theory as a whole.  Computers were used as early as 1953 to
classify semigroups of low order up to isomorphism and anti-isomorphism: all 4
semigroups of order 2 and all 18 semigroups of order 3 were classified by Tamura
in 1953 \cite{tamura_1953}; Forsythe followed in 1955 with the 126 semigroups of
order 4 \cite{order_4}; and the following year Motzkin and Selfridge found the
1160 semigroups of order 5 \cite{order_5, jurgensen_1977}.  Despite these early
successes, however, the
theory of computing with semigroups developed more slowly than with groups, and
no package emerged for semigroups on the scale of the group algorithms in
\GAP{}.  However, there has been considerable interest in computational
semigroup theory in recent years, and increasingly there do exist algorithms for
semigroups, as well as software packages implementing them, such as
\textsf{Semigroup for Windows} \cite{sgpwin}, \Semigroupe{} \cite{semigroupe},
\libsemigroups{} \cite{libsemigroups}, and the \GAP{} packages \Semigroups{}
\cite{semigroups}, \smallsemi{} \cite{smallsemi} and \kbmag{} \cite{kbmag}.

In the same way that semigroup theory borrows results from group theory,
computational semigroup theory often borrows algorithms from computational group
theory.  The Todd--Coxeter algorithm, for instance, was originally designed to
calculate a subgroup's cosets inside a group -- but with a few changes, it can
be used to find the elements of a finitely presented semigroup, or the classes
of a congruence on such a semigroup, as we will see in this thesis.  We also
borrow from computational group theory by using properties of a semigroup's
subgroups.  For example, when computing the linked triples on a completely
simple semigroup, we require algorithms from computational group theory to find
all the normal subgroups of a given maximal subgroup.  In this way,
computational group theory is not just a subset, but a key part, of
computational semigroup theory.

This thesis deals primarily with the congruences on a semigroup.  A semigroup's
congruences describe its homomorphic kernels and images -- that is, the ways in
which the semigroup can be mapped onto another semigroup while preserving the
operation.  In this way, a semigroup's congruences serve the same function as a
group's normal subgroups, or a ring's two-sided ideals, and are of as much
interest in semigroup theory as those structures are in their respective fields.
Classifying the congruences of important semigroups has long been a major
activity in semigroup theory.  Some important early examples are the full
transformation monoid $\T_n$ by Mal$'$cev \cite{malcev_1952}, the symmetric
inverse monoid $\I_n$ by Liber \cite{liber_1953}, and the partial transformation
monoid $\PT_n$ by Shutov \cite{shutov_1988} -- and more recently, the direct
product of any pair of these by Ara{\'u}jo, Bentz and Gomes \cite{bentz_congs}.
A host of other semigroups have had their congruences classified over the years
-- for example, from \cite{fernandes_2000, lisbon_ii, lisbon_i} we know the
congruences on various monoids of partial transformations restricted to elements
that preserve or reverse the order or orientation of the set being acted on.
Moving away from partial transformations, Mal$'$cev also classified the
congruences on the semigroup $F_n$ of $n \times n$ matrices with entries from a
field $F$ \cite{malcev_matrices} -- and the congruences on a direct product
$F_m \times F_n$ of these are also classified in \cite{bentz_congs}.  While
these examples are by no means an exhaustive list, there remain many important
semigroups whose congruences have not yet been classified.  Finding the
congruences on various other semigroups will form the majority of Part
\ref{part:results} of this thesis.

The computational theory of semigroup congruences is also a young field.
Algorithms for computing information about a given congruence have existed in
the \GAP{} library for many years, but many of them lack sophistication and may
take an unreasonably long time to return results about semigroups with more than
about 25000 elements.  We can improve on these algorithms in two different ways.
Firstly we can consider alternative structures that correspond to a semigroup's
congruences.  For instance, in group theory we consider a group's normal
subgroups rather than studying its congruences directly; in the same way, we can
study a completely simple semigroup's linked triples, or an inverse semigroup's
kernel--trace pairs, in place of their congruences, and thus we can often
produce the answers to computational questions more quickly than by using direct
methods.  This approach forms the basis of the present author's previous works
\cite{mtorpey_pre_msc, mtorpey_msc}, which are expanded upon in this
thesis. Secondly, in the more general cases where no such alternative structure
exists, we may still make improvements on existing algorithms by applying new
algorithms that are not currently used for congruences, such as the
Todd--Coxeter coset enumeration procedure, or the Knuth--Bendix completion
process.  We will take both approaches in Part \ref{part:algorithms} of this
thesis, presenting new congruence algorithms, as well as showing ways in which
existing algorithms can be successfully adapted for congruence-related purposes.

After introducing some preliminary theory, this thesis is divided into two broad
parts.  Part \ref{part:algorithms} discusses the computational theory of
congruences, and presents algorithms that can be used to answer
congruence-related questions.  Part \ref{part:results} shows some results that
have been obtained using these algorithms, as well as results that have been
proven by hand using computational output as a starting point.

Chapter \ref{chap:intro} acts as an introduction to this document, providing the
preliminary knowledge which is required to understand the material in the rest
of the thesis.  It is mostly concerned with ideas from semigroup theory such as
Green's relations, generators, congruences and presentations.  It also
introduces several important types of elements that form semigroups, such as
partial transformations and bipartitions.  Later, some computational issues such
as algorithms and decidability are covered, as well as some algorithms that are
used later on.

Chapter \ref{chap:pairs} presents a new way of computing with congruences
defined by generating pairs.  The method presented uses parallel computation to
run a variety of algorithms -- including Todd--Coxeter, Knuth--Bendix, and an
unsophisticated algorithm known as \textit{pair orbit enumeration} -- that test
whether a given pair lies in a congruence, given its generating pairs.  This
approach takes advantage of the different algorithms' abilities to return an
answer quickly in various cases, in each case exhibiting run-times close to the
minimum of all the different algorithms.  Modifications for left and right
congruences are explained, and different versions are described for finitely
presented semigroups and for \textit{concrete} semigroups (finite semigroups of
transformations or other similar elements).  This approach was implemented in
\libsemigroups{} \cite{libsemigroups}, and the results of benchmarking tests on
that implementation are shown near the end of the chapter.

Chapter \ref{chap:converting} concerns the various ways of representing a
congruence, other than as a set of pairs.  Five possible representations are
described -- generating pairs, normal subgroups, linked triples, kernel--trace
pairs, and ideals -- along with explanations of the precise semigroups and
congruences to which they apply.  Algorithms are given for converting from one
representation to another, so that a computational algebra system may quickly
convert a congruence specified in a certain way to the most efficient
representation, and thus answer questions about the congruence in as short a
time as possible.  All twenty of the possible conversions between the five
representations are considered: many of these are currently implemented in
\Semigroups{} \cite{semigroups}, some having never been described before; others
are trivial or unnecessary; and two remain open problems (converting from
kernel--trace to generating pairs, and from generating pairs to an ideal).  See
Table \ref{tab:converting} for a summary of all the conversions described in the
chapter.

Chapter \ref{chap:lattice} presents an algorithm for computing the entire
congruence lattice of a finite semigroup.  This algorithm is rather rudimentary,
but various shortcuts and improvements are described to try to reduce the
computational work required, where possible.  A version of this algorithm is
implemented in \Semigroups{}, and thus takes advantage of the methods described
in Chapters \ref{chap:pairs} and \ref{chap:converting}.  This makes it possible
to compute the lattices of any sufficiently small semigroup in a reasonable
time; a brief analysis is included of the sizes of semigroup and lattice which
are feasible, along with some visual examples of lattices that have been
computed using this method.

Chapter \ref{chap:motzkin} considers the Motzkin monoid $\Mot_n$, the monoid of
all planar bipartitions of degree $n$ with blocks of size no greater than $2$.
An investigation into the congruence lattice of this monoid was initiated by
computational experiments, using the method described in Chapter
\ref{chap:lattice}, and resulted in a complete classification of the congruence
lattice of $\Mot_n$ for arbitrary $n$, along with generating pairs for each
congruence.  This classification, originally published in \cite{ourpaper}, is
shown here with the kind permission of my co-authors.  Other important monoids
of bipartitions are also considered, and their congruence lattices classified.

Chapter \ref{chap:other} completes this thesis by presenting some other results
obtained or precipitated by computational experiments in the \Semigroups{}
package.  Firstly, we consider the congruences on the principal factors of the
full transformation monoid $\T_n$ and some other related monoids, classifying
them for arbitrary $n$.  Secondly, we consider the number of congruences that
exist on an arbitrary finite semigroup.  We give some upper and lower bounds for
this number based on a semigroup's size, and we present two conjectures about
the second-largest number of congruences a finite semigroup can have.  To
support these conjectures, we also show some computational evidence produced
with the aid of \smallsemi{} \cite{smallsemi}.  Finally, we present some
findings from an exhaustive classification of the congruences on all $853303$
semigroups of size no larger than $7$, up to isomorphism and anti-isomorphism.

At the end of this document we provide an index of the various terms that are
used.  We also provide a list of notation, with a brief description of what each
mathematical symbol means.  In both of these, each entry has a reference to the
page on which the term or symbol is first defined.  In the original digital
version of this document, all citations and numbered references act as
hyperlinks, allowing the reader to click them to be redirected to the
appropriate location.
