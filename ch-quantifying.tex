\chapter{Quantifying doubling and uniform perfectness}
\label{chap:quantifying}


\section{Introduction}
\label{sec:intro-quantifying}


The goal of this chapter is to quantify how doubling a measure is via the upper regularity dimension and uniformly perfect through the lower regularity dimension. This will start with a simple relationship between these notions which will then help understand a classic technical result regarding the measure of balls. The notion of quasisymmetric invariant will then be studied by calculating bounds for the regularity dimensions of pushforward measures under such maps with respect to the dimensions of the original measures. Finally a relation between the lower regularity dimension and a regularity of measures property recently introduced in Diophantine approximation will be discussed. This leads to a more quantitative statement of a recent theorem in the Diophantine approximation on Kleinian groups.





\section{Results}\label{ch-quantifying:sec:results}

We start by establishing some well known facts and introducing our results. Section \ref{ch-quantifying:sec:equivalence} will study the link between the regularity dimensions and the doubling constants and constants of uniform perfectness and these notions will help to quantify a well known technical proposition in Section \ref{ch-quantifying:sec:quantifying}. Quasisymmetric homeomorphisms will be introduced in Section \ref{ch-quantifying:sec:quasi} with the goal of understanding how the regularity dimensions can be changed under the action of such maps. Finally Section \ref{ch-quantifying:sec:diophantine} will be spent discussing a result from Diophantine approximation and relating it to our work on the regularity dimensions.


\subsection{Quantifying doubling and uniform perfectness}\label{ch-quantifying:sec:equivalence}

Recall from the introduction, that a measure is doubling if and only if it has finite upper regularity dimension; similarly for uniform perfectness and the lower regularity dimension. Repeatedly, we have stated that this means the regularity dimensions quantify doubling and uniform perfectness without further elaboration. This can be formalised so the upper regularity dimension of a doubling measure can be stated explicitly as a function of the doubling constants; similarly for the lower regularity dimension with respect to the constants of uniform perfectness when this dimension is stictly positive. 

\begin{theorem}\label{ch-quantifying:equivalence}
Let $\mu$ be a doubling measure fully supported on a metric space $X$ with doubling constants $C(\theta)$, then $$\urdim \mu = \inf_{\theta > 1}\frac{\log C(\theta)}{\log \theta}.$$ Similarly if $\mu$ is a uniformly perfect measure with constants of uniform perfectness $K(\theta)$ then $$\lrdim \mu = \sup_{\theta > 1} \frac{\log K(\theta)}{\log \theta}.$$
\end{theorem}


Unfortunately these formulae are supremums and infimums and so the regularity dimensions are not even always attained for any $\theta$. This means we cannot restate results regarding the doubling constants with solely the upper regularity dimension.

We consider an example which helps illustrates this result. Let $F$ be the middle third Cantor set and $\mu$ the self-similar measure fully supported on $F$ with associated probability vector $\left\{ 1/4, 3/4 \right\}$, as introduced in the previous chapter. We know the upper regularity dimension is $\log 4 / \log 3$ whilst the lower regularity dimension is $\log(4/3) / \log 3$. Start by just considering balls centred on the origin, of radius $3^{-k}$ for some $k \in \mathbb{N}$ and with $\theta = 3^{l}$ for any $l \in \mathbb{N}$. Due to the structure of $F$ it is clear that 
\[
\mu(B(0, 3^{-k})) = 4^{l} \mu(B(0, 3^{-(k+l)}))
\]
and so 
\[
C(3^{l}) \ge 4^l.
\] 
If this was an equality we would have attained the upper regularity dimension for a finite $\theta$, however the above simple analysis only covers the point $x=0$ and scales $r = 3^{-k}$. Heuristically, as the set is self-similar, changing $x$ or $r$ shouldn't change the doubling constant $C(3^{l})$ much, but slight perturbations are likely sufficient to increase the doubling constant enough so that the upper regularity dimension is not attained for $\theta = 3^l$. A similar behaviour can be observed at $x = 1$ for the lower regularity dimension. It would be interesting to study this example more and even consider other ranges of $\theta$, perhaps using renewal theory or other techniques.


\begin{question}
Is there always a doubling constant with respect to a specific $\theta$ which recovers the upper regularity dimension, both for specific cases such as self-similar measures and in full generality?
\end{question}


\subsection{Quantifying an example of Heinonen}\label{ch-quantifying:sec:quantifying}


When studying doubling measures a technical proposition is often employed to truly benefit from the regularity of these measures. Simply put, a doubling measure on a uniformly perfect space is also a uniformly perfect measure. This implies the following important bounds. Say $\mu$ is a doubling measure on a uniformly perfect, bounded space $X$, then there exists constants $0< \lambda_1, \lambda_2 < \infty$ and $0 < t \le s < \infty$ such that for any $x \in X$ and $0 < r$
\[
\lambda_1 r^s \le \mu(B(X,r)) \le \lambda_2 r^t.
\]
It is not clear where this was first stated, but the standard reference \cite{heinonen} provides this result as an example (\cite[Exercise 13.1]{heinonen}) without a proof. 

This result opens a number of questions. Notably, how close to zero can the lower regularity dimension be in this setting? Clearly the lower regularity dimension of $\mu$ must depend on the lower dimension of $X$, since the lower dimension is an upper bound, see \cite{bylund}. But in principle this does not bound the lower regularity dimension from below in any meaningful way. 

One could ask if there exists a fixed uniformly perfect space and a sequence of doubling measures on that space which all have the same upper regularity dimension but whose lower regularity dimensions can be made as small as possible. A cursory check of some standard examples such as self-similar sets and measures implies this might not be feasible, Theorem \ref{ch-upper-reg:selfsimilar} of the previous chapter is helpful here. Heuristically, by keeping the upper regularity dimensions of self-similar measures fixed with a goal of decreasing the lower regularity dimension, we must increase a probability from the associated probability vector which is not the minimum of the vector. As this is a probability vector, the sum of the elements is always one so any increase must be accompanied by a decrease in at least one other element. If the lower regularity dimension is to approach zero in the limit a probability must tend towards one and thus all other probabilities must be lowered towards zero. This will inevitably increase the upper regularity dimension after some number of steps, breaking the assumption.

\begin{figure}[h]
    \centering
    \begin{tikzpicture}[scale=0.8]
		\coordinate (O) at (0,0);
		\coordinate (A) at (\Width,0);
		\coordinate (B) at (\Width,\Height);
		\coordinate (C) at (0,\Height);
		\coordinate (C1) at (0,\Height*2/4);
		\coordinate (C2) at (0,\Height*3/4);
		\coordinate (C3) at (\Width/3,\Height*2/4);
		\coordinate (C4) at (\Width/3,\Height*3/4);
		\coordinate (A1) at (\Width*2/3,\Height/4);
		\coordinate (A2) at (\Width*2/3,\Height*2/4);
		\coordinate (A3) at (\Width,\Height/4);
		\coordinate (A4) at (\Width,\Height*2/4);
		\coordinate (D1) at (\Width*2/3,\Height*3/4);
		\coordinate (D2) at (\Width*2/3,\Height);
		\coordinate (D3) at (\Width,\Height*3/4);
		\coordinate (D4) at (\Width,\Height);
		\draw[black] (O) -- (A) -- (B) -- (C) -- cycle;% Bottom Face
		\draw[black,fill=gray!70] (C1) -- (C2) -- (C4) -- (C3) -- cycle;
		\draw (C1) rectangle (C4) node[pos=.5] {\LARGE$\varepsilon$};
		\draw[black,fill=gray!70] (A1) -- (A2) -- (A4) -- (A3) -- cycle;% Bottom Face
		\draw (A1) rectangle (A4) node[pos=.5] {\LARGE$ 1 -  2\varepsilon$};
		\draw[black,fill=gray!70] (D1) -- (D2) -- (D4) -- (D3) -- cycle;% Bottom Face
		\draw (D1) rectangle (D4) node[pos=.5] {\LARGE $\varepsilon$};
\end{tikzpicture}
    \caption{Example of a self-affine set and measure which also demonstrates how the increase in probability of the lower right map will force the remaining probabilities to zero}
    \label{fig:pathos-ex}
\end{figure}


This lends credibility to the conjecture that the example of Heinonen implies a lower bound to the lower regularity dimension as a function of the upper regularity dimension. We wish to quantitatively show this result. More precisely, given a measure $\mu$ of fixed upper regularity dimension $s$ on a space $X$ of fixed lower dimension $l$, can we bound the lower regularity dimension of $\mu$ as a function of $s$ and $l$? The following does not quite answer this question as it returns a function of the doubling and uniform perfectness constants. However, as we will discuss, this is closer to the desired solution than it appears.

\begin{proposition}\label{ch-quantifying:result-heinonen}
	If $X$ is uniformly perfect and $\mu$ is a doubling, fully supported measure on $X$ then $\mu$ is uniformly perfect. In particular if $X$ is $K$-uniformly perfect and $\mu$ is doubling with doubling constants $C(\theta)$ then  
	$$\lrdim \mu \ge \frac{\ln(1-C(8K)^{-1})}{\ln(4K)} .$$  
\end{proposition}


Combining this result with Theorem \ref{ch-quantifying:equivalence}, it follows that a doubling measure on a uniformly perfect space must have lower regularity dimension bounded below by a function of the upper regularity dimension and the constant of uniform perfectness. However, as eluded to previously, Theorem \ref{ch-quantifying:equivalence} is concerned only with the infimum over all $\theta$ and the formula in Proposition 2.1 relies on a specific $C(\theta)$. Thus we cannot, in general, find an exact formula linking the lower regularity dimension and the upper regularity dimension in our setting.


We finish this section with a brief discussion of the sharpness of this result. Assuming positive lower dimension of the space is required here as it is simple to construct spaces of zero lower dimension but finite Assouad dimension. In such a setting there must exist a measure of upper regularity dimension close to the Assouad dimension of the space and so doubling. But any measure on this space cannot be uniformly perfect as the lower regularity dimension is a lower bound to the lower dimension. A trivial such example would be the set of points $\left\{ 1/n : n \in \mathbb{N} \right\}$ with the Euclidean metric. This set is know to have zero lower dimension but full Assouad dimension; doubling measures on this space were explicitly constructed in the previous chapter. On the other hand we will find examples of measures on uniformly perfect spaces which are neither doubling nor uniformly perfect, see Chapter 4 for further information. Thus the obtained bound is justified in demanding both the doubling constants of $\mu$ and the constant of uniform perfectness of the set.

There are many examples of uniformly perfect measures, even on doubling spaces, that are not doubling so we cannot interchange the two notions and obtain an analogous result.  For instance, one can take a self-similar set with overlaps, this is a doubling space. Numerous uniformly perfect measures exist on such a space and are not doubling, as can be seen in \cite{hare-troscheit} where all self-similar measures were shown to be uniformly perfect. Thus a uniformly perfect measure on a space of positive, finite lower and Assouad dimension need not be doubling.




\subsection{Regularity dimensions under quasisymmetric homeomorphisms}\label{ch-quantifying:sec:quasi}


Quasisymmetric homeomorphisms are a generalisation of bi-Lipschitz maps, preserving relative sizes but not necessarily global size. These were first introduced in \cite{ahlfors-beurling, tukia-vaisala}. In the Euclidean setting quasisymmetric homeomorphisms are equivalent to the often studied quasiconformal homeomorphisms. Recall that $d_X(\cdot,\cdot)$ denotes the metric on the space $X$; tracking which space we are working in will be important in this section. A homeomorphism $f\colon X \rightarrow Y$ is an \textit{$\eta$-quasisymmetric homeomorphism (or map)} if there is a homeomorphism $\eta \colon [0,\infty) \rightarrow [0,\infty)$ such that 
\[
d_X( x , a ) \le t \, d_X( x , b )
\]
implies 
\[
d_Y( f(x) , f(a) ) \le \eta(t) \, d_Y ( f(x) , f(b) )
\]
for all $x,a,b \in X$ and for all $t>0$.

Equivalently there exists a homeomorphism $\eta$ as above such that 
\[
\frac{d_Y(f(x),f(y))}{d_Y(f(x),f(z))} \le \eta \left(\frac{d_X(x,y)}{d_X(x,z)} \right)
\]
for any distinct points $x,y,z \in X$. We will sometimes talk about quasisymmetric homeomorphisms without specifying $\eta$, this should be read as there exists a homeomorphism $\eta$ for which the function is an $\eta$-quasisymmetric homeomorphism but the exact form of $\eta$ is not important at that time and is omitted. Here $\eta$ is clearly not unique for a given quasisymmetric homeomorphism. 


A property of particular interest to us is that doubling and uniform perfectness of spaces are quasisymmetric invariants. This can be quantified, so there are bounds on the Assouad and lower dimensions of images of spaces under quasisymmetric embeddings. Formally for a set $F$ and a quasisymmetric map $f$ we have
\[
\alpha \Assouad F \le \Assouad f(F) \le \frac{1}{\alpha} \Assouad F
\]
where $\alpha$ is some constant intrinsic to the map $f$ which will be introduced shortly. This is stated as an exercise in \cite{heinonen} due to Tyson. Note that these estimates are not necessarily true when $f$ is H\"older, see \cite{luk}

In fact, there is even a bijection between doubling measures and quasisymmetric homeomorphisms on the real line. Formally every doubling measure $\mu$ on $\mathbb{R}$ is associated with a quasisymmetric map $f(x) = \int_{0}^{x}d\mu$ and every quasisymmetric $f \colon \mathbb{R} \rightarrow \mathbb{R}$ induces a doubling measure $\mu_f( \cdot ) = \mathcal{L}(f(\cdot)) $ where $\mathcal{L}$ is Lebesgue measure. The classic book \cite{heinonen} contains further details and many other interesting properties of these maps that we will not touch upon in this thesis.

We wish to know if the same holds for doubling and uniformly perfect measures. In particular we will study \textit{pushforward measures} under quasisymmetric homeomorphisms. Given a measure $\mu$ on a space $X$ and $f$ a measurable map from $X$ to some space $Y$, the pushforward measure of $\mu$ under $f$ is denoted $f_*\mu$ and is defined by
\nomenclature[mu3]{$f_*\mu$}{pushforward measure of $\mu$ under map $f$}
\[
f_*\mu (A) = \mu(f^{-1}(A))
\]
for any measurable subset $A$ of $Y$, where $f^{-1}(A) = \left\{x \in X \colon f(x) \in A \right\}$. 

To avoid having trivial upper and lower regularity dimensions of $\mu$ it is reasonable to assume that $X$ is doubling and uniformly perfect. This then lets us employ the following theorem. 

\begin{theorem}[{\cite[Theorem 13.11]{heinonen}}]
	A quasisymmetric homeomorphism $f$ of a uniformly perfect space $X$ is $\eta$-quasisymmetric with $\eta$ of the form
	\[
	\eta(t) = c_\eta \max\left\{t^\alpha, t^{1/\alpha}\right\},
	\]
	where $c_\eta \ge 1 $ and $\alpha \in (0,1]$ depend only on $f$ and $X$.
\end{theorem}
For clarity we will often write $\eta_{\alpha}$ to indicate the homeomorphism $\eta$ associated with the constant $\alpha$ as described here. Section 3 of \cite{tukia-vaisala} proves this result and explicitly calculates $\alpha$. 

For such quasisymmetric homeomorphisms we see that doubling and uniform perfectness of measures are also invariants, mirroring the geometric setting.

\begin{theorem}\label{ch-quantifying:quaisymm-thm}
	Let $X$ be a uniformly perfect space and $\mu$ be doubling on $X$. When $f$ is an $\eta_\alpha$-quasisymmetric homeomorphism the following bounds hold
	\[
	\alpha \, \urdim \mu \le \urdim f_{*}\mu \le \urdim \mu/\alpha
	\]
	and 
	\[
	\alpha \, \lrdim \mu \le \lrdim f_{*}\mu \le \lrdim \mu/\alpha
	\]
	where $f_{*}\mu = \mu \circ f^{-1}$ is the pushforward of $\mu$.
\end{theorem}



\subsection{Uniformly perfect and weakly absolutely $\alpha$-decaying measures}\label{ch-quantifying:sec:diophantine}

A property that has appeared recently in Diophantine approximation is the notion of \textit{weakly absolutely $\alpha$-decaying}. This was first introduced in \cite{beres-sanju-al} following the previous uses of friendly measures by \cite{friendly} and quasi-decaying measures by \cite{decaying1, decaying2}. A measure $\mu$ is weakly absolutely $\alpha$-decaying for some $\alpha > 0$ when there exists constants $C, R_0 >0$ such that for all $\varepsilon > 0$
\[
\mu(B(x,\varepsilon R)) \le C \varepsilon^{\alpha} \mu(B(x,R))
\]
for all $x \in X$ and $R<R_0$.

This property has some resemblance to the ideas of doubling and uniformly perfect measures. We show the following.
\begin{proposition}\label{ch-quantifying:equiv-diophantine}
	If a measure $\mu$ has positive lower regularity dimension then 
	\[
	\lrdim \mu = \sup\left\{\alpha \colon \mu \text{ is weakly absolutely $\alpha$-decaying}\right\}.
	\]
\end{proposition}

This result actually leads to an equivalent but hopefully more applicable statement of Theorem 2 in \cite{beres-sanju-al} regarding Diophantine approximaton. We start with a bit of background to motivate this concept.

A well know result of Dirichlet states that for any $x \in \mathbb{R}$ and $N \in \mathbb{N}$, there exists integers $p, q \in \mathbb{Z}$ such that $1 \le q \le N $ and 
\[
d_{\mathbb{R}}\left( x , \frac{p}{q} \right) \le \frac{1}{qN}.
\]
Here and throughout $d_{\mathbb{R}}$ denotes the Euclidean metric on the real line. This result can be though of as quantifying how well the rationals are dense in the reals. A natural extension is to ask what happens when the approximating term $1/qN$ is replaced by another decreasing function $\psi \colon [0,\infty) \rightarrow [0,\infty)$. The goal is then to study the size of the points which satisfy the previous inequality with respect to $\psi$ infinitely often. Formally we call such points $\psi$-well approximable and denote them by 
\[
W(\psi) = \left\{x \in \mathbb{R} \colon d_{\mathbb{R}}\left( x , \frac{p}{q} \right) \le \psi(q) \text{ for infinitely many } (p,q) \in \mathbb{Z} \times \mathbb{N}  \right\}.
\]
Note that usually these results are stated in $\mathbb{R}^d$ but for now we restrict our exposition to the real line. The well known Khintchine's Theorem provides the size of $W(\psi)$ in the context of Lebesgue measure.
\begin{theorem}
    Let $\mathcal{L}$ be Lebesgue measure. Then
    \[
    \mathcal{L}(W(\psi)) = \begin{cases} 
      0 & \textup{if }   \sum_{r=1}^{\infty}\psi(r)r < \infty, \\
      1 & \textup{if }   \sum_{r=1}^{\infty}\psi(r)r = \infty.
   \end{cases}
    \]
\end{theorem}

Analogously there has been much work understanding approximation results in non-Euclidean settings. In particular a version of Dirichlet's Theorem was proved by Patterson \cite{patterson} for finitely generated Kleinian groups acting on the unit disc model of hyperbolic space. 

In this section we work in $(d+1)$-dimensional hyperbolic space using the ball model $\mathbb{D}^{d+1} = \left\{ z \in \mathbb{R}^{d+1} \colon d_{\mathbb{R}^{d+1}}(\mathbf{0},z) < 1 \right\}$ with the hyperbolic metric $d_{\mathbb{H}}$ derived by $ds = 2 \lvert dz \rvert / (1 - \lvert z \rvert^2)$. Here $\mathbf{0} = (0,\ldots, 0) \in \mathbb{R}^{d+1}$ is just the origin. Consider the group of orientation-preserving M{\"o}bius transformations of the unit ball $\mathbb{D}^{d+1}$, denoted M{\"o}b$(\mathbb{D}^{d+1})$. Let $G$ be a geometrically finite, discrete subgroup of M{\"o}b$(\mathbb{D}^{d+1})$, such groups are called Kleinian groups. A group of M{\"o}bius transformations is geometrically finite if it has a fundamental domain with finitely many sides. 

We will be interested in studying the limit set of $G$, denoted $\Lambda$, which is the set of limit points in the unit sphere $S^d$ of any orbit of $G$ in $\mathbb{D}^{d+1}$. Non-elementary Kleinian groups have uncountable limit sets, otherwise the limit set contains either 0, 1 or 2 points which does not provide interesting geometric behaviour. 

A M{\"o}bius transformation is parabolic if and only if it has a unique fixed point in the boundary of hyperbolic space $S^d$ whilst a hyperbolic element will fix exactly two points in $S^d$. For any $g \in G$ let $L_g = d_{\mathbb{R}^{d+1}}(\mathbf{0}, g'(\mathbf{0}))^{-1}$, where $d_{\mathbb{R}^{d+1}}(\mathbf{0}, g'(\mathbf{0})) = 1 - d_{\mathbb{R}^{d+1}}(\mathbf{0}, g(\mathbf{0})) ^2$ is the Euclidean conformal dilation of $g$ at the origin. $L_g$ will tell us how far a group element will move a point in hyperbolic space towards the boundary and is the equivalent of $q$ in the rational setting. A version of Dirichlet's Theorem due to Patterson is then the following.

\begin{theorem}
Let $G$ be a non-elementary, geometrically finite Kleinian group without parabolic elements and let $\left\{ \eta, \eta' \right\}$ be the pair of fixed points of a hyperbolic element of $G$. Then there is a constant $c > 0$ with the following property: for all $\xi \in \Lambda$, $N > 1$, there exist
$y \in \left\{ \eta, \eta'\right\}$, $g \in G$ so that
\[
d_{\mathbb{R}^{d+1}}(\xi, g(y)) \le \frac{c}{N} \quad \textup{ and } \quad L_g \le N.
\]
\end{theorem}

As in the Euclidean setting we can study points which are well-approximable, this time with respect to a point $y \in \Lambda$ which will usually be taken as a fixed point of a hyperbolic or parabolic element of $G$. Formally, for a Kleinian group $G$, let $\psi \colon [0,\infty) \rightarrow [0,\infty)$ be a decreasing function and define
\[
W_y(\psi) = \left\{ \xi \in \Lambda \colon d_{\mathbb{R}^{d+1}}(\xi, g(y)) \le \psi(L_g) \text{ for infinitely many } g \in G \right\}.
\]
As $W_y(\psi)$ is a subset of the limit set we cannot use Lebesgue measure to study the size of the well-approximable numbers. The well studied Patterson-Sullivan measure $\mu_{PS}$ is the natural answer to this as non-elementary geometrically finite Kleinian groups are known to carry this measure and in this setting it is an atomless conformal ergodic Borel probability measure. An analogue of Khintchine's Theorem in this setting can be show through the work of \cite{patterson, stratmann, stratmann-velani, beres-dick-velani}; in its final form the original 1-dimensional version of Khintchine's Theorem can even be recovered. To avoid overcomplicating this section we omit the full statement of the hyperbolic analogue.

Returning to the classical setting one last time, a theory of Diophantine approximation on manifolds has developped. For instance one can ask what is the size of $\mathcal{M} \cap W(\psi)$ for some submanifold $\mathcal{M}$ of $\mathbb{R}^d$. Size here would be the normalised $k$-dimensional Lebesgue measure on $\mathcal{M}$ where $\mathcal{M}$ is a $k$-dimensional manifold. The question then becomes which conditions must be imposed on $\mathcal{M}$ so that an analogue of Khintchine's can be recovered.

Similarly Beresnevich et al. \cite{beres-sanju-al} asked a similar question for subsets of limit sets in the hyperbolic setting, which is the focus of this section. Let $K$ be a subset of the limit set $\Lambda$ which supports a non-atomic probability measure $\mu$. The desired regularity property of $K$ in this setting is that $\mu$ is wealy absolutely $\alpha$-decaying. The following result can then be shown.

\begin{theorem}[{\cite[Theorem 2]{beres-sanju-al}}]
	Let $G$ be a nonelementary, geometrically finite Kleinian group and let $y$ be a parabolic fixed point of $G$, if there are any, and a hyperbolic fixed point otherwise. Fix $\alpha > 0$, and let $K$ be a compact subset of $\Lambda$ equipped with a weakly absolutely $\alpha$-decaying meausre $\mu$. Then
	\[
	\mu(K\cap W_y(\psi)) = 0 \quad if \quad \sum_{r=1}^\infty r^{\alpha-1} \psi(r)^{\alpha} < \infty.
	\]
\end{theorem}

Using Proposition \ref{ch-quantifying:equiv-diophantine} we can deduce a new version of this theorem.

\begin{theorem}
	Assume $G$ and $y$ are as above. Let $K$ be a compact subset of $\Lambda$ with lower dimension equal to $s > 0$. For any $0 < \alpha < s$, there exists a weakly absolutely $\alpha$-decaying measure $\mu$ on $K$ and 
	\[
	\mu(K\cap W_y(\psi)) = 0 \quad if \quad \sum_{r=1}^\infty r^{\alpha-1} \psi(r)^{\alpha} < \infty.
	\]
	In particular, if  $\sum_{r=1}^\infty r^{s-1} \psi(r)^{s} < \infty$ then any weakly absolutely $\alpha$-decaying measure on $K$ is such that $\mu(K\cap W_y(\psi)) = 0$.
\end{theorem}

The existence of weakly absolutely $\alpha$-decaying measure $\mu$ for any $0< \alpha < s$ is due to recent work of Suomala \cite[Theorem 4.1]{suomala}.

An advantage of writing the theorem with respect to the lower dimension of subsets of the limit set is that the lower dimension of limit sets were calculated in \cite{fraser2}. Therefore, given a limit set, we can quickly check if there will be measures that are weakly absolutely $\alpha$-decaying such that the sum in the theorem converges for the given $\alpha$. In \cite{fraser2}, Fraser also calculated the regularity dimensions of Patterson-Sullivan measures, providing us with explicit measures that could be used in the theorem for suitable subsets, as the upper and lower regularity dimensions of Patterson-Sullivan measures are strictly positive and finite. 

Theorem 11 follows from Proposition 7 and Theorem 10. As such we do not provide a proof here, for the interested reader the proof of Theorem 10 in \cite{beres-sanju-al} is very accessible. For further information on Kleinian groups and Diophantine approximation we refer the reader to \cite{beres-sanju-al, fraser2} and the references therein.

Weakly absolutely decaying measures were the correct measures to consider in the setting of limit sets of Kleinian groups whereas friendly measures were used in the context of subsets of Euclidean space. It would be a natural extension to study the links between friendly measures and the regularity dimensions, especially given that one of the conditions for a measure to be friendly is that a subset of full measure is `locally doubling', see \cite{friendly} for the complete definition.  

\begin{question}
Is there a relation between the regularity dimensions and other notions of regularity of measures used in Diophantine approximation?
\end{question}



\section{Proofs}


This proof section will be broken into several subsections that are mostly independent of each other but the notation will remain consistent throughout. In Section \ref{ch-upper-reg:doublingproof} we cover the ideas found in Section \ref{ch-quantifying:sec:equivalence} relating the regularity dimensions with their associted regularity constants. In Section \ref{ch-quantifying:sec:proof-heinonen} we show Proposition \ref{ch-quantifying:result-heinonen}. Section \ref{ch-quantifying:proof-quasi} is dedicated to the study of quasisymmetric homemorphisms and Theorem \ref{ch-quantifying:quaisymm-thm}. Finally in Section \ref{ch-quantifying:diophantine-proof} a short proof of Proposition \ref{ch-quantifying:equiv-diophantine} is provided.


\subsection{Proof of equivalence of doubling and finite upper regularity dimension} \label{ch-upper-reg:doublingproof}


We start by proving the link between the upper regularity dimension and the doubling constants. Recall the notion of doubling implies the existence of constants $C(\theta)$; we wish to pick the optimal constant. Given $\theta > 1$ let
\[
C(\theta) = \sup \left\{ \frac{\mu(B(x,R))}{\mu(B(x,R/\theta))} \ : \ x \in \text{supp}(\mu), R>0 \right\} \geq 1.
\]

Assume that a measure $\mu$ on a space $X$ is doubling and let $\theta > 1$.  Therefore, for all $x \in \text{supp}(\mu)$ and $R>0$
\[
\frac{\mu(B(x,R))}{\mu(B(x,R/\theta))} \le C(\theta).
\]
Fix  $0< r < R$ and  define $k$ to be the unique integer such that $R\theta^{-k} > r \ge R\theta^{-k-1}$. Then by telescoping we obtain
\begin{align*}
    \frac{\mu(B(x,R))}{\mu(B(x,r))} = \frac{\mu(B(x,R))}{\mu(B(x,\theta^{-1} R))} &\times \frac{\mu(B(x,\theta^{-1} R))}{\mu(B(x,\theta^{-2} R))} \times\cdots \\
    &\times \frac{\mu(B(x,\theta^{-k} R))}{\mu(B(x,\theta^{-k-1}R))} \times \frac{\mu(B(x,\theta^{-k-1}R))}{\mu(B(x,r))}.
\end{align*}
Thus 
\begin{align*}
    \frac{\mu(B(x,R))}{\mu(B(x,r))}& \le C(\theta)^{k+1}  \frac{\mu(B(x,\theta^{-k-1}R))}{\mu(B(x,r))} \\
    &\le C(\theta)^{\log(R/r) / \log \theta+1} = C(\theta) \left(\frac{R}{r} \right)^{\log C(\theta) / \log \theta}
\end{align*}
and hence $\r \leq \log C(\theta) / \log \theta < \infty$, giving the upper bound.


To obtain a lower bound on the upper regularity dimension, it suffices to find, for $s = \inf\frac{\log C(\theta)}{\log \theta}$, a sequence of $x\in X$ and $0<r<R$, with $R/r \rightarrow \infty$, such that 
\[
\frac{\mu(B(x,R))}{\mu(B(x,r))} \ge \left(\frac{R}{r}\right)^s.
\]
	
	
From the definition of doubling we know that $\mu(B(x,\theta r) ) \le C(\theta) \mu(B(x,r))$ for all $x,r, \theta$. Fixing $\theta$ we pick $C(\theta)$ to be reasonably sharp in the sense that there exists at least one pair of $x,r$ such that $\mu(B(x,\theta r) ) \ge \frac{1}{2}C(\theta) \mu(B(x,r))$. Choosing $C(\theta)$ to be optimal as above would ensure this for instance.
	
Recall $s = \inf_{\theta > 1}\frac{\log C(\theta)}{\log \theta}$. To choose our sequence of $x,r$ and $R$ we simply pick any sequence of increasing $\theta$. Then from our choice of $C(\theta)$, the pair $x$ and $r$ are the pair obtained above. $R$ is then fixed by $R = \theta r$. Finally, due to the choice of $s$,
\[
\frac{\mu(B(x,R))}{\mu(B(x,r))} \ge \frac{1}{2} C(\theta)  \ge \frac{1}{2}\theta^s = \frac{1}{2}\left(\frac{R}{r} \right)^s,
\]
completing the proof for the upper regularity dimension.
	

The lower regularity dimension result follows similarly. Recall a measure $\mu$ is uniformly perfect if and only if it has positive lower regularity dimension and thus there exists constants $K(\theta)$ for all $\theta > 1$ such that 
\[
K(\theta) =  \inf \left\{ \frac{\mu(B(x,R))}{\mu(B(x,R/\theta))} \ : \ x \in \text{supp}(\mu), R>0 \right\} \geq 1.
\]

For all $x \in \text{supp}(\mu)$ and $0<r<R$, fix $k$ to be the unique integer such that $R \theta^{-k} > r \ge R \theta^{-k-1}$. Then by the above argument
\begin{align*}
\frac{\mu(B(x,R))}{\mu(B(x,r))} &= \frac{\mu(B(x,R))}{\mu(B(x,\theta^{-1}R))} \times \cdots \times \frac{\mu(B(x,\theta^{-k}R))}{\mu(B(x,r))} \\
&\ge K(\theta)^k \frac{\mu(B(x,\theta^{-k}R))}{\mu(B(x,r))} \ge K(\theta)^k.
\end{align*}
Approximating $k$ yields
\[
\frac{\mu(B(x,R))}{\mu(B(x,r))} \ge K(\theta)^{\log(R/r)\log(\theta) - 1} = K(\theta)^{-1} \left(\frac{R}{r}\right)^{\log K(\theta)/ \log \theta}
\]
so $\lrdim \mu \ge \log K(\theta) / \log \theta$ for any $\theta > 1$ as desired. 

For the upper bound, we again find a sequence of $x\in X$ and $0<r<R$ which will attain the dimension. Recall, for a fixed $\theta > 1$, we can choose $x,r$ such that $\mu(B(x,\theta r)) \le \frac{1}{2}K(\theta)\mu(B(x,r))$. 

Let $t = \sup_{\theta > 1} \frac{\log K(\theta)}{\log \theta}$ and choose any sequence of strictly increasing $\theta > 1$. This gives us sequences of $x,r$ for which $K(\theta)$ satisfies the above condition and then pick $R = \theta r$. Thus
\[
\frac{\mu(B(x,R))}{\mu(B(x,r))} \le \frac{1}{2} K(\theta) \le \frac{1}{2} \theta^t = \frac{1}{2} \left(\frac{R}{r} \right)^t
\]
completing the proof.



\subsection{Quantifying an example of Heinonen}\label{ch-quantifying:sec:proof-heinonen}



Let $X$ and $\mu$ be as in the statement of the proposition with constant of uniform perfectness $K$ and doubling constants $C(\theta)$. We will rework the proof found in \cite[lemma 3.1]{anti1}, paying careful attention to the constants in play. Note there is another proof in \cite[Lemma 4.5]{eino-pablo} which could lead to different bounds, but we do not pursue this here.
	
To start, a technical result is required. Proposition B.4.7, in \cite{gromov}, states that in our setting there exists a constant $a \in (0,1)$ such that 
\[
\mu(B(x,aR)) \le (1-a) \mu(B(x,R))
\]
for any $x,R$. We start by determining $a$ as a function of our known constants. 
	
For any $x\in X$ and $R>0$, as $X$ is uniformly perfect, there exists $y \in X$ such that $$\frac{R}{2K} \le d(x,y) \le \frac{R}{2}.$$ This choice of $y$ ensures that $B(x,\frac{R}{4K}) \cap B(y,\frac{R}{4K}) = \emptyset $ and $B(x,\frac{R}{4K}) \cup B(y,\frac{R}{4K}) \subseteq B(x,R)$. Thus
\begin{align*}
\mu(B(x,R/(4K))) &\le \mu(B(x,R)) - \mu(B(y, R/(4K)))\\
& \le \mu(B(x,R)) - \frac{1}{C(8K)}\mu(B(y,2R)) \\
& \le \mu(B(x,R)) - \frac{1}{C(8K)}\mu(B(x,R)) \\
& = (1-C(8K)^{-1}) \mu(B(x,R)).
\end{align*}
Recall $C(8K)$ is the doubling constant of $\mu$ where $\theta = 8K$. By iterating this construction we obtain
\[
\mu(B(x,R/(4K)^n)) \le (1-C(8K)^{-1})^n \mu(B(x,R))
\]
for any $n\in \mathbb{N}$, as desired.
	
Returning to the actual question, fix $x\in X$, $0 < r < R$ and choose $n\in \mathbb{N}$ such that $(4K)^{-n-1}R < r \le (4K)^{-n}R$ so that $B(x,r) \subseteq B(x,R/(4K)^{n})$. Then
\begin{align*}
\frac{\mu(B(x,R))}{\mu(B(x,r))} \ge& \frac{\mu(B(x,R))}{(1-C(8K)^{-1})^n\mu(B(x,R))} \\
& \ge (1-C(8K)^{-1})^{\frac{-\ln(R/r)}{\ln(4K)} + 1}\\
& = (1-C(8K)^{-1})\left(\frac{R}{r}\right)^{\frac{\ln(1-C(8K)^{-1})}{\ln(4K)}}
\end{align*}
as desired.
	
Note that in the proof of \cite{gromov} one should use the optimal doubling and uniform perfectness constants to obtain the best bound possible, however the result itself is likely not sharp.






\subsection{Regularity dimensions under quasisymmetric homeomorphism}\label{ch-quantifying:proof-quasi}


Whilst Theorem 13.11 of \cite{heinonen} is the key ingredient in the proof of Theorem \ref{ch-quantifying:quaisymm-thm}, the following proposition which can be found in \cite{heinonen} is also required.

\begin{proposition}[{\cite[Proposition 10.6]{heinonen}}]
	When a quasisymmetric homeomorphism $f\colon X \rightarrow Y$ is $\eta$-quasisymmetric, its inverse $f^{-1}$ is an $\eta'$-quasisymmetric homeomorphism with $\eta'$ given by  $\eta'(t) = 1/\eta^{-1}(1/t)$ for $t>0$.
\end{proposition}

It is then clear that a quasisymmetric homeomorphism $f$ on a uniformly perfect space is associated with a homeomorphism $\eta(t) = c_\eta\max\left\{t^\alpha, t^{1/\alpha}\right\}$ and $f^{-1}$ is also a quasisymmetric homeomorphism associated with the function $1/\eta^{-1}(1/t) \le c_\eta^{1/\alpha} \max\left\{t^\alpha, t^{1/\alpha} \right\}$. Note that the homeomorphism $\eta'(t) = c_\eta^{1/\alpha} \max\left\{t^\alpha, t^{1/\alpha} \right\}$ is not exactly $1/\eta^{-1}(1/t)$, but, as it is a upper bound to the desired function, $f^{-1}$ will be an $\eta'$-quasisymmetric homeomorphism.



We start by stating some technical calculations which will be relevant for both the upper and lower regularity dimensions. Let $y\in Y$ and $0<r<R$. Since $Y$ is uniformly perfect, we can find $z_1,z_2$ such that $z_1\in B_Y(y,KR) \setminus B_Y(y,R)$ and $z_2 \in B_Y(y,Kr) \setminus B_Y(y,r)$. Without loss of generality, choose $ Kr < R $ so that $d_Y(y,z_1) > d_Y(y,z_2)$,  this will be required to use the exact formula for $\eta$.
	
	
Choose any point $a \in B_Y(y,R)$. From our choice of $z_1$, it is clear that $d_Y( y , a ) \le d_Y(y , z_1)$. Thus, as $f$ is quasisymmetric 
$$d_X( f^{-1}(y) , f^{-1}(a) ) \le \eta(1)  d_X(f^{-1}(y) , f^{-1}(z_1), $$
and so 
\[
f^{-1}(B_Y(y,R)) \subseteq B_X(f^{-1}(y),\eta(1)d_X(f^{-1}(y), f^{-1}(z_1)).
\]
Similarly, choosing $a \in B_Y(y,R) \setminus B_Y(y,R/K)$, we have $d_Y( y , a ) \ge d_Y( y , z_1 )/K^2$ and so $$d_X( f^{-1}(y) , f^{-1}(a) ) \ge \eta^{-1}(K^2) d_X( f^{-1}(y) , f^{-1}(z_1) ).$$
Hence 
\[
f^{-1}(B_Y(y,R)) \supseteq B_X(f^{-1}(y),\eta^{-1}(K^2)d_X(f^{-1}(y),f^{-1}(z_1))).
\]
Similar statements clearly hold for $r$ with $z_2$. 

We can now prove the upper bound for the upper regularity dimension. For any $\varepsilon > 0$,
\begin{align*}
\frac{\mu(f^{-1}(B_Y(y,R)))}{\mu(f^{-1}(B_Y(y,r)))} &\le \frac{\mu(B_X(f^{-1}(y),\eta(1)d_X(f^{-1}(y), f^{-1}(z_1)) ))}{\mu(B_X(f^{-1}(y),\eta^{-1}(K^2)d_X(f^{-1}(y),f^{-1}(z_2)) ))} \\
& \le C_{\varepsilon}\left( \frac{\eta(1)d_X(f^{-1}(y),f^{-1}(z_1))}{\eta^{-1}(K^2)d_X(f^{-1}(y),f^{-1}(z_2))}\right)^{\urdim \mu + \varepsilon} \\
& \le C_{\varepsilon}\left(c_\eta^{\alpha} \eta(1)/\eta^{-1}(K^2)\right)^{\urdim \mu + \varepsilon} \left(\frac{d_Y(y,z_1)}{d_Y(y,z_2)} \right)^{(\urdim \mu + \varepsilon)  / \alpha} \\
& \le C_{\varepsilon}\left(c_\eta^{\alpha} \eta(1)/\eta^{-1}(K^2)\right)^{\urdim \mu + \varepsilon} \left(\frac{KR}{r} \right)^{(\urdim \mu +\varepsilon)/ \alpha},
\end{align*}
where $C_{\varepsilon}$ is the constant from the definition of the upper regularity dimension of $\mu$ with respect to $\varepsilon$. As $\varepsilon$ is arbitrarily chosen this completes the upper bound, that is $\urdim f_*\mu \le \urdim \mu / \alpha$.
	
For the lower bound recall $f^{-1}$ is an $\eta'$-quasisymmetric homeomorphism where $\eta'$ is of the form $\eta_\alpha$ with the same $\alpha$ as in $\eta$. Thus the measure $f^{-1}_* (f_*\mu)$ is the pushforward measure of $f_*\mu$ under the quasisymmetric map $f^{-1}$. Therefore the above bound also holds for this new measure and so $\urdim f^{-1}_* (f_*\mu) \le \urdim f_* \mu / \alpha$. Since $f$ is a homeomorphism this yields $\alpha\, \urdim \mu \le \urdim f_* \mu$ as expected. 


The lower regularity dimension can be shown in much the same way, we start with the lower bound. Let $\varepsilon > 0$,
\begin{align*}
\frac{\mu(f^{-1}(B_Y(y,R)))}{\mu(f^{-1}(B_Y(y,r)))} &\ge \frac{\mu(B_X(f^{-1}(y),\eta^{-1}(K^2)d_X(f^{-1}(y),f^{-1}(z_1)) ))}{\mu(B_X(f^{-1}(y),\eta(1)d_X(f^{-1}(y), f^{-1}(z_2)) ))} \\
& \ge D_{\varepsilon}\left( \frac{\eta^{-1}(K^2)d_X(f^{-1}(y),f^{-1}(z_1))}{\eta(1)d_X(f^{-1}(y),f^{-1}(z_2))}\right)^{\lrdim \mu - \varepsilon} \\
& \ge D_{\varepsilon}\left(\frac{1}{c_\eta^{\alpha}} \eta^{-1}(K^2)/\eta(1)\right)^{\lrdim \mu - \varepsilon} \left(\frac{d_Y(y,z_1)}{d_Y(y,z_2)} \right)^{\alpha \,(\lrdim \mu - \varepsilon)} \\
& \ge D_{\varepsilon}\left(\frac{1}{c_\eta^{\alpha}} \eta^{-1}(K^2)/\eta(1)\right)^{\lrdim \mu - \varepsilon} \left(\frac{R}{K r} \right)^{\alpha \,(\lrdim \mu - \varepsilon)},
\end{align*}
where $D_\varepsilon$ is the constant stemming from the lower regularity dimension of $\mu$ as a function of $\varepsilon$. Letting $\varepsilon$ tend to zero finishes the proof of $\lrdim f_*\mu \ge \alpha \, \lrdim \mu$.

For the upper bound, we again argue that $f_*^{-1}(f_*\mu) = \mu$ giving $\lrdim f_*^{-1}(f_* \mu) \ge \alpha \, \lrdim f_*^{-1} \mu$. Hence, $\lrdim f_* \mu \le \lrdim \mu / \alpha$.











\subsection{Uniformly perfect and weakly absolutely $\alpha$-decaying measures}\label{ch-quantifying:diophantine-proof}



	If $\mu$ is weakly absolutely $\alpha$-decaying then $\mu(B(x,\varepsilon R)) \le C \varepsilon^{\alpha} \mu(B(x,R))$ for any $x\in X$ and $\varepsilon, R >0$. Thus
	\[
	\frac{\mu(B(x,R))}{\mu(B(x,\varepsilon R))} \ge \frac{1}{C} \left(\frac{R}{\varepsilon R} \right)^{\alpha}
	\]
	and so $\lrdim \mu \ge \alpha$.
	
	For the other direction, assume $\lrdim \mu = t$. Then for any $\delta > 0$, there exists $C' > 0$ such that for all $x\in X$ and $R>r>0$ 
	\begin{align*}
	\frac{\mu(B(x,R))}{\mu(B(x,r))} &\ge C' \left( \frac{R}{r}\right)^{t - \delta}.
	\end{align*}
	Given $\varepsilon \in (0,1)$ and $R > 0$, choose $r = \varepsilon R$. Inserting this value of $r$ into the above yields 
	\[\mu(B(x,R)) \ge C' \left(\frac{R}{\varepsilon R}\right)^{t-\delta} \mu(B(x,\varepsilon R)).
	\]
	Hence
	\[
	\mu(B(x,\varepsilon R)) \le \frac{1}{C'} \varepsilon^{t - \delta} \mu(B(x,R))  
	\]
	and so $\mu$ is $(t-\delta)$-decaying.




